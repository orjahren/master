\DocumentMetadata{lang=en, pdfversion=1.7, pdfstandard=A-2u}
\documentclass[UKenglish]{uiomasterthesis}  %% ... or norsk or nynorsk or USenglish
\usepackage[utf8]{inputenc}                 %% ... or latin1
\usepackage[T1]{url}\urlstyle{sf}
\usepackage{subfig} % for figurer som tar mindre plass
\usepackage{babel, csquotes, graphicx, textcomp, frontpage/uiomasterfp, varioref}
\usepackage[backend=biber,style=numeric-comp]{biblatex}
\usepackage[hidelinks, hypertexnames=false]{hyperref}

% Custom packages
\usepackage{tcolorbox} % drawing boxes
\usepackage{glossaries} % acronyms
\usepackage{cleveref} % in-document references
\usepackage{siunitx} % \num
\usepackage{paralist} % inparenum
\usepackage{svg} % for SVG content on the front page
\usepackage{epigraph} % for quotes
\usepackage{listings} % for code blocks

% Latex-related imports
% Farger/stil fra https://www.overleaf.com/learn/latex/Code_listing 
\definecolor{codegreen}{rgb}{0,0.6,0}
\definecolor{codegray}{rgb}{0.5,0.5,0.5}
\definecolor{codepurple}{rgb}{0.58,0,0.82}
\definecolor{backcolour}{rgb}{0.95,0.95,0.92}

\usepackage[rgb]{xcolor}%Nødvendig for UU-sjekk ifbm code listings. Uten dette blander den inn CMYK.

\lstdefinestyle{mystyle}{
  backgroundcolor=\color{backcolour},
  commentstyle=\color{codegreen},
  keywordstyle=\color{magenta},
  numberstyle=\tiny\color{codegray},
  stringstyle=\color{codepurple},
  basicstyle=\ttfamily\tiny,
  breakatwhitespace=false,
  breaklines=true,
  captionpos=b,
  keepspaces=true,
  numbers=left,
  numbersep=5pt,
  showspaces=false,
  showstringspaces=false,
  showtabs=false,
  tabsize=2
}

\lstset{style=mystyle}

% For å rendre diffs, fra https://tex.stackexchange.com/a/106129
\usepackage[svgnames]{xcolor}
\definecolor{diffstart}{named}{Grey}
\definecolor{diffincl}{named}{Green}
\definecolor{diffrem}{named}{OrangeRed}

\lstdefinelanguage{diff}{
  %basicstyle=\ttfamily\small,
  morecomment=[f][\color{diffstart}]{@@},
  morecomment=[f][\color{diffincl}]{+\ },
  morecomment=[f][\color{diffrem}]{-\ },
}

% Skrur av links i glossary-entries, se #5.
\glsdisablehyper{}
\makeglossaries{}

% Set globals
\setglossarystyle{altlist}
\setdefaultenum{(1)}{(a)}{i.}{A.}
\interfootnotelinepenalty=10000

\setlength\epigraphwidth{.8\textwidth}
\setlength\epigraphrule{0pt}
\epigraphnoindent{}

\microtypecontext{spacing=nonfrench}

% Header styles
\pagestyle{fancy}
\fancyhf{} % clear all header and footer fields

% Header: Chapter name on even pages, Section name on odd pages
\fancyhead[LE]{\small\textsf{\leftmark}}  % Left header even pages
\fancyhead[RO]{\small\textsf{\rightmark}} % Right header odd pages

% Page number: right on even, left on odd
\fancyhead[RE]{\small\textsf{\thepage}}   % Right header even pages
\fancyhead[LO]{\small\textsf{\thepage}}   % Left header odd pages

% Optional: thin line under header
\renewcommand{\headrulewidth}{0.4pt}

% Footer: nothing (can add if you want)
\renewcommand{\footrulewidth}{0pt}

\setlength{\headsep}{20pt} % Denne angir avstanden mellom page header og section header
\setlength{\parskip}{1em}
\newcommand{\hefe}{\textsc{Hefe}}

\NewDocumentCommand{\Nref}{s m}{%
  \hyperref[#2]{%
    \nameref*{#2}%
    \IfBooleanF{#1}{%
      \textsuperscript{\ttfamily\tiny\(\,\to\,\)p.\labelcpageref*{#2}}%
    }%
  }%
}

\RenewDocumentCommand{\Cref}{s m}{%
  \hyperref[#2]{%
    \nameCref{#2}~\labelcref*{#2}%
    \IfBooleanF{#1}{%
      \textsuperscript{\ttfamily\tiny\(\,\to\,\)p.\labelcpageref*{#2}}%
    }%
  }%
}

%% Save the original link commands
\NewCommandCopy{\hrefold}{\href}
\NewCommandCopy{\hyperlinkold}{\hyperlink}
\NewCommandCopy{\urlold}{\url}

%% Use standard colored links for URLs, hyperlinks, references, and citations (breakable, no boxes)
\RenewDocumentCommand{\href}{O{} m m}{\hrefold{#2}{#3}}
\RenewDocumentCommand{\url}{O{} m}{\urlold{#2}}
\RenewDocumentCommand{\hyperlink}{O{} m m}{\hyperlinkold{#2}{#3}}

% Fra UiO-malen 
\providecommand{\frontmatter}{\cleardoublepage \pagenumbering{roman}}
\providecommand{\mainmatter}{\cleardoublepage \pagenumbering{arabic}}
\providecommand{\backmatter}{\cleardoublepage}

% For å kunne ha 2 abstracts uten at de lager ny side
\patchcmd{\abstract}{\titlepage}{\thispagestyle{empty}}{}{}
\patchcmd{\endabstract}{\endtitlepage}{\clearpage}{}{}
\newacronym[plural=LLMs, longplural={Large Language Models}]{llm}{LLM}{Large Language Model}
\newacronym[plural=ADSs, longplural={Autonomous driving systems}]{ads}{ADS}{Autonomous driving system}
\newacronym[plural=APIs, longplural={Application programming interfaces}]{api}{API}{Application programming interface}
\newacronym{jit}{JIT}{Just-In-Time}
\newacronym{ml}{ML}{Machine learning}
\newacronym{dl}{DL}{Deep learning}
\newacronym{csv}{CSV}{Comma separated values}
\newacronym{gui}{GUI}{Graphical user interface}
\newacronym{cv}{CV}{Computer Vision}
\newacronym{dsc}{DSC}{Driving situation complexity}
\newacronym{icl}{ICL}{In-context learning}
\newacronym{cot}{CoT}{Chain-of-Thought}
\newacronym{rl}{RL}{Reinforcement learning}
\addbibresource{refs.bib}


% NOTE: Denne er tom og populeres ved runtime av CI for å tagge dens output.
\usepackage[
    text={First draft, \today},    % Text of the watermark
    fontsize=36pt,   % Font size of the watermark
    color=red,      % Color of the watermark
    angle=0,
    scale=0.5,          % Scale factor for the watermark
    vanchor=t,
    vpos=0.01\paperheight,
]{draftwatermark}

\title{Can large language models make our roads safer?}
\subtitle{Utilising large language models to decrease driveability in autonomous driving system simulator scenarios}
\author{Oliver Ruste Jahren}               

\begin{document}
\uiomasterfp[dept={Department of Informatics},
  program={Informatics: Programming and Systems architecture},
  supervisors={Shaukat Ali \and Karoline Nylænder},
  info={Simula Research Laboratory},
  date={Fall 2025},
  image={frontpage/header.jpg},
  color={orange},
  long]

\frontmatter{}
\begin{abstract}
  % Here come 3--6 sentences describing your thesis.
  \acrfullpl{ads} rely on extensive testing in order to verify their operational safety. But due to
  their nature of being able to operate in any unseen environment with arbitrary external actors,
  the number of potential scenarios is infinite. It is therefore important to obtain the least
  driveable scenarios for simulator testing ahead of real world deployment. We therefore propose
  applying \acrlongpl{llm} to \acrlong{ads} scenario files to decrease their driveability, exposing
  potential underlying issues in the \acrshort{ads} being tested in advance of it happening during
  real world operation, avoiding causing severe damage to its operator and/or other external actors.
\end{abstract}

% TODO: Disse 2 sammendragene er veldig forskjellige. Er det OK?

\begin{xabstract}[Sammendrag]
  %Here comes the abstract in a different language.
  For å kunne få selvkjørende biler ut på veiene, må vi være sikre på at de er trygge. Trygge både
  for seg selv, sjåføren, og andre traffikanter. Men det ligger i en bils natur at den skal kunne
  brukes overalt, med alle mulige folk inne i bildet. Derfor er det teoretisk umulig å forutse alle
  mulige situasjoner og teste disse i forkant. På bakgrunn av dette fremmer vi i dette arbeidet en
  metode for å ta ibruk KI til å gjøre dagens testscenarioer \textit{mer utrygge} enn hva de
  allerede er. Å teste med disse forespeiles å ville kunne avdekke potensielle underliggende feil i
  bilens systemer, slik at de kan rettes før den volder skade ute i verden.
\end{xabstract}

\tableofcontents{}
\listoffigures{}                            %% (omit if none)
% \listoftables{}                             %% (omit if none)
\lstlistoflistings{}

\begin{preface}
  Here comes your preface, including acknowledgments and thanks.
\end{preface}

\mainmatter{}
\part{Introduction}
\chapter{Introduction}\label{chp:introduction}

\epigraph{A problem well stated is a problem half solved.}{Charles F. Kettering}

This chapter presents the motivation and problem statement of the thesis,
condensing the the problem statement to a set of formalized research questions.
Finally, we present the structure of the thesis with an outline of the topics of
each chapter.

\section{Motivation}

Conventional cars are ubiquitous in society~\cite[1]{thorgersen2021why}. Whether for freight
trafficking or for humans, cars have great flexibility with their ability to go wherever without
requiring tailored infrastructure such as railway tracks. They do, however, have one major weak
point --- the human driver~\cite[67]{Riener2010}. For this reason, industry and academia have put
forward efforts to enhancing cars with \acrfull{ads} capabilities. By empowering humans with
autonomous vehicles, it is expected that traffic safety and efficiency will increase along with
comfort as well as enabling the development of several other new transportation methods~\cite[1-2,
    1]{Eichberger2020, weiADSAdoption}.

Due to the critical safety situation of operating a car, it is essential that \acrshort{ads} are
thoroughly tested before they are deployed so that they are verified to be sufficiently safe and
capable of handling the situations in which they may typically end
up~\cite[1]{surveyLLMScenarioBasedTesting}.
But due to the complicated nature of the typical
\acrshort{ads} operating environment, coming up with exhaustive system test solutions is near
impossible~\cite[52]{DeepScenario}. For this reason, we want a way of testing the system that is
capable of pushing the \acrshort{ads} to its limits such that we can measure its performance and see
if it is capable of handling critical scenarios~\cite{yao2025agentsllm}.

Existing methods for testing \acrshortpl{ads} typically rely on driving billions of miles, but this
incurs high cost, and is time-intensive~\cite[1]{surveyLLMScenarioBasedTesting}. To address some
of these concerns, \acrshort{ads} simulators have been utilised. But it's not the case that we will
\emph{know} that the \acrshort{ads} is safe after it has driven $x$ kilometres on roads or
$y$ kilometres in a simulator -- we will never be able to predict all possible operating
situations in advance~\cite[1]{leahy2024grandchallenge}.
Therefore, we need to challenge the \acrshort{ads} as much as possible when testing it. Having it
take on a set of low-driveability test scenarios in advance of real-world operations,
we can be more confident in the correctness of our \acrshort{ads} if it is able to complete the
scenario. By making the scenario \emph{more complex} -- less driveable -- for
the \acrshort{ads}, our confidence in it will increase if it is able to complete the scenario. And
if it fails at executing the more complex scenario, we will potentially have uncovered an
underlying issue in the \acrshort{ads} that we did not know about so that we can fix it before it
causes harm in the real world.

Having an existing repository of \acrshort{ads} simulator scenarios, we wish to improve
them in such a way that they are less driveable and more challenging for the \acrshort{ads}.
\acrfullpl{llm} have demonstrated great capabilities of context learning and emergent
abilities~\cite[1]{LLM4AD}, which begs the question of their applicability for \acrshort{ads}
testing. We therefore ask: Can these existing test scenarios be made less driveable by applying
\acrshort{llm} technology to them?

\section{Problem description}\label{sec:problemDescription}

Traditional techniques for obtaining \acrshort{ads} scenarios rely on
\begin{inparaenum}
    \item highly skilled manual labour, or
    \item automated generation~\cite[1]{yao2025agentsllm}.
\end{inparaenum}
The prior incurs a significant cost, and is a major limitation in obtaining a large number of good
scenarios. The latter incurs a \emph{distributional shift} from the original scenarios, which can
undermine the validity of using them~\cite[1]{yao2025agentsllm}.

Moreover, even if we were to imagine a world in which we had infinite \begin{inparaenum}
    \item time and
    \item money, \end{inparaenum} we would not be able to successfully account for every possible
scenario.
There will always be more, unforeseen permutations of actors and actions. This is a reality
we need to deal with~\cite[1]{leahy2024grandchallenge}.
One possible measure of remedying with this, could be to \textit{decrease} the
driveability of our existing scenarios. Decreasing the driveability is not the same as suddenly
having access to the infinite set of possible scenarios, but it is reasonable to infer that being
able to \textit{test} the \acrshort{ads} (in a simulator) on these enhanced low-driveability
scenarios will leave it better fit for encountering other low-driveability scenarios in the wild
during operation. Having access to such a set of less driveable scenarios will allow \acrshort{ads}
operators to test their \acrshort{ads} that they assume to be working, and see if it still is able
to handle all these more challenging scenarios. If it is not, they will have gained a meaningful
insight into the workings of their \acrshort{ads} and can take action to remedy the fault before it
causes harm in the real world. And if the \acrshort{ads} \emph{does} still work, they can be more
confident in their system.

Finally, edge cases can be a major issue for \acrshort{ads} adaptation. The \textit{tail problem} as
it is known in the \acrfull{ml} field posits that \acrshort{ml} tasks are faced with a long tail of
unseen cases. We can map these unseen cases, to our unseen \acrshort{ads} scenarios. Because of
this, an \acrshort{ads} can be at risk of encountering an unseen edge case scenario during operation
-- something for which it might never have been tested~\cite[1]{yao2025agentsllm}.
Arguing that the \acrshort{ads} would probably crash simply due to it finding itself in an unseen
scenario is not logical. But it is important to keep in mind that the end we are pursing in the
broader adaption of \acrshort{ads}, is increased safety and efficiency on our roads. Not
sufficiently testing the \acrshort{ads} before deploying it would not serve our goal of increasing
road safety -- it would be a gamble with human lives. Not all edge cases are relevant for all
contexts and environments. As such, \acrshort{ads} system developers could employ \acrshortpl{llm}
to obtain more variants of a certain scenario that they wish to test, if they were to only be in possession of
a similar form. Say for example that they wished to obtain a set of scenarios in which
certain properties are met, or certain situations occur. 

Based on the problem description above, these are the formal research
questions of the thesis:

% Define each RQ as a macro so we can re-use them in the discussion. DRY.
\newcommand{\rqdecreasedriveability}{Can \acrlongpl{llm} be used to decrease the driveability of \acrlong{ads} simulator scenarios?}
\newcommand{\rqnohuman}{Is it feasible to employ \acrshortpl{llm} for obtaining unseen scenarios for \acrshort{ads} testing without human intervention?}

\begin{quote}
    \begin{questions}
        \item \rqdecreasedriveability
        \defrq{decrease-driveability}{\rqdecreasedriveability}\label{rq:decrease-driveability}
        \item \rqnohuman
        \defrq{no-human}{\rqnohuman}\label{rq:no-human}
    \end{questions}
\end{quote}

\newpage % Legger inn en newpage siden den nye siden uansett blir brukt opp.
% Dette ser bedre ut.

\section{Thesis overview}

Following this Introduction chapter, the thesis is structured as follows:

\begin{itemize}
    \item \Cref{chp:background} introduces key concepts related to \acrshortpl{ads} and \acrshortpl{llm}.
    \item \Cref{chp:relatedWorkAndLitReview} reviews the current state of research in the field and discusses related applied works and lessons learned.
    \item \Cref{chp:solutionProposal} details our proposed solution and its technical aspects.
    \item \Cref{chp:experiments} describes the experimental setup used to evaluate the solution.
    \item \Cref{chp:results} presents the findings from the experiments.
    \item \Cref{chp:discussion} analyses and contextualizes the results.
    \item \Cref{chp:furtherWork} suggests directions for future research.
    \item \Cref{chp:conclusion} summarizes the main contributions and findings.
\end{itemize}

Two appendices are included: \Nref{sec:fileDiffs} and \Nref{sec:errorMessages}.

\chapter{Background}\label{chp:background}

\epigraph{The limits of my language mean the limits of my world.}{Wittgenstein}

This chapter will give an introduction to \acrshortpl{ads} and \acrshortpl{llm},
laying the foundation for understanding the motivation of the project more
in-depth, such that the appeal of the later solution proposal is more clear. 


% \subsection{Autonomous systems}
% 
% Autonomous systems are systems that are capable of changing their behaviour in
% response to unanticipated events during
% operation~\cite[368]{watson2005autonomous}. Autonomous systems exist in several
% different forms and they are typically tailored for their specific operating
% environments. Some systems operate in the maritime domain, some in the air, and
% others on the ground in the terrestrial operating domain~\cite[369-370]{watson2005autonomous}.

\section{Autonomous driving systems (ADSs)}

\acrfull{ads} are systems that enable automotive vehicles to drive autonomously. Due to the typical
operating scenarios of a car it is pivotal that the \acrlong{ads} maintain a high safety standard. A
common way to assert safety is to use simulator based testing~\cite[1]{DeepScenario}.

\subsection{Autonomous driving system driveability}\label{sec:adsDrivability}

\emph{Driveability} is a high-level estimator of the overall driving
condition of an \acrshort{ads}, derived from several lower-level sources~\cite[3140]{safeToDrive}.
It can be used to refer to various aspects of a scene.
\citeauthor{safeToDrive} discuss the concept further, using the  scene definition
of~\citeauthor{scenes} as outlined in  \Cref{sec:adsSimConcepts}, they describe
how driveability can refer both to \begin{inparaenum}
    \item road conditions, and
    \item human driver performance.
\end{inparaenum}
\citeauthor{safeToDrive} go on to give an overview of how driveability
can be used to refer to a \begin{inparaenum}\setcounter{enumi}{2}
    \item \textit{driveability map} which divides a map into
    cells indicating where the \acrshort{ads} expects that it will be able to go, and
    \item \textit{object driveability}, which refers to the classification of physical objects in
    the environment that the \acrshort{ads} expects that it can run over without causing damage to
    the ego vehicle~\cite[3135-3136]{safeToDrive}.
\end{inparaenum}

The main method for assessing the driveability of a scene comes form assessing the environment of
the scene. Factors such as \begin{inparaenum}
    \item weather,
    \item traffic flow,
    \item road condition, and
    \item obstacles \end{inparaenum} all play into this. The \acrshort{ads} infers information from
observation~\cite[3136]{safeToDrive}.

They continue to give an overview of various \textit{driveability factors} and their associated
difficulties, using a a split between \textit{explicit} and \textit{implicit} factors.

\emph{Explicit driveability factors} will typically include factors such as \emph{Extreme
    weather} such as \begin{inparaenum}
    \item fog,
    \item heavy rain,
    \item snow,
\end{inparaenum}
all serving to impair road visibility and causing increased difficulties for vision-based tasks such
as road detection and object tracking~\cite[3136-3137]{safeToDrive}. \emph{Illumination} also
poses various challenges for typical \acrshort{ads} tasks as a typical \acrshort{ads} will be
required to operate in a plethora of scenes with varying degrees of illumination depending on
factors such as time of day and location (e.g. if the \acrshort{ads} is operation in a dimly lit
tunnel)~\cite[3137]{safeToDrive}. The authors highlight how low illumination may serve as an
advantage for the \acrshort{ads} as this allows for using the head lights of other vehicles as a
feature for detecting them, whereas it make pedestrian detection significantly more
challenging~\cite[3137]{safeToDrive}. \emph{Road geometry} is another external factor, satisfying
our natural intuition that \textit{intersections} and \textit{roundabouts} are more difficult to
drive through than straight highways~\cite[3137]{safeToDrive}.

\emph{Implicit driveability factors} consist of behaviours and intent of other road users
interacting with the autonomous car~\cite[3138]{safeToDrive}. This includes the actions of other
vehicles such as their \begin{inparaenum}
    \item overtaking,
    \item lane changing,
    \item rear-ending,
    \item speeding, and
    \item failure to obey traffic laws \end{inparaenum}.~\citeauthor{safeToDrive} call these factors
\emph{vehicle behaviours}~\cite[3138]{safeToDrive}. Furthermore, \emph{pedestrian behaviours}
are also taken into account, noting how pedestrians can sometimes
\begin{inparaenum}\setcounter{enumi}{5}
    \item cross the road,
    \item be inattentive, or
    \item fail to comply with the traffic law \end{inparaenum}\cite[3138]{safeToDrive}. They go on
to describe the \emph{driver behaviour} of other drivers pointing out how
\begin{inparaenum}\setcounter{enumi}{8}
    \item distraction, and
    \item drowsiness \end{inparaenum} can be factors that cause accidents even for
\acrshort{ads}-enhanced vehicles due to the other, manual, cars
interfering with their operation~\cite[3138-3139]{safeToDrive}. Lastly
\emph{motorcyclist/bicyclist behaviours} cause their own source of implicit driveability factors:
The models and methods developed for analysing the group's behaviour are far more limited than other
groups of road users~\cite[3139]{safeToDrive}.~\citeauthor{safeToDrive} theorise that this comes
down to the lack of available datasets that capture and label the trajectories and behaviours
of motorcyclists and bicyclists~\cite[3139]{safeToDrive}, causing potential issues for any
\acrshort{ads} that wishes to operate in a shared traffic environment with this group.

\section{Autonomous driving system testing}

Testing is essential for assuring \acrlong{ads} operative safety~\cite[163]{ADTestingReview16}.
Several methods for testing exist, testing various aspects of the \acrlong{ads}. An \acrshort{ads}
typically exists of several modules, all working together and handling different aspect of the
\acrlong{ads}.

\citeauthor{ADTestingReview16} outline several typical architectures for \acrshort{ads} testing,
drawing on traditional software testing traditions outlining how \textit{software testing} can be
used alongside more specialized \acrshort{ads} testing techniques such as \textit{simulation
    testing} and \textit{X-in-the-loop testing}~\cite[163-164]{ADTestingReview16}.


\subsection{Autonomous driving system testing metrics}\label{sec:adsMetrics}

When evaluating \acrshort{ads} testing, several metrics can be used. What metric to use will depend
on what the relevant test is measuring.

Building on what we have learnt about driveability (\Cref{sec:adsDrivability}),
we take after \citeauthor{safeToDrive} and review three metrics for
quantifying driveability: \begin{inparaenum}
    \item scene driveability,
    \item collision-based risk, and
    \item behaviour-based risk. Finally, we also investigate the mertric
    \item jerk.
\end{inparaenum}

\emph{Scene driveability} refers to how easy a scene is for an
\acrshort{ads} to navigate, and the \textit{scene driveability score} refers to
how likely the \acrlong{ads} is to fail at traversing the
scene~\cite[3140]{safeToDrive}. It is typically found through and end-to-end
approach. Note how this is a metric for \textit{scenes}, without taking into
account the performance of any specific \acrshort{ads}.

\emph{Collision-based risk} comes in two kinds - \begin{inparaenum}
    \item binary risk indicator, and
    \item probabilistic risk indicator.
\end{inparaenum} \citeauthor{safeToDrive} posit that the prior, binary metric, indicates whether a
collision will happen in the near future in a binary `either-or' sense, whereas the latter yields a
probability calculated based on current states, event, choice of hypothesis, future states and
damage~\cite[3140]{safeToDrive}.

\emph{Behaviour-based risk} estimation also represents a binary classification problem wherein
nominal behaviours are learnt from data, and then dangerous behaviours are detected on that. This
requires a definition of `nominal behaviour', which is typically defined on on acceptable speeds,
traffic roles, location semantics, weather conditions and/or the level of fatigue of the
driver~\cite[3140]{safeToDrive}. Furthermore \citeauthor{safeToDrive} describe how this metric also
allows more than one \acrshort{ads} to be labelled as `conflicting' or `not
conflicting'~\cite[3140]{safeToDrive}, representing a ruling on their compatibility. Finally, they
note how behaviour-based risk assessment typically focuses on driver behaviours, not taking into
account other actors in the scene such as pedestrians or cyclists.

Furthermore, \emph{jerk} is a metric that renders the change of vehicle acceleration with respect to
time. It has been used been used as a measure of the smoothness or abruptness of a movement in many
domains such as the trajectory planning of the human arm and industrial
robots~\cite[126]{fengJerk17}. Jerk has also been shown to relate to a driver’s
physiological feelings of ride comfort~\cite[126]{fengJerk17}, giving it a clear relation to our
previously stated definition of driveability.~\citeauthor{fengJerk17} go on to posit that a goal of
driving should be to minimize the jerk, as it both is both \begin{inparaenum}
    \item linked to comfort, and
    \item detection of safety-critical events
\end{inparaenum}\cite[126]{fengJerk17}.

\subsection{The complexities of ADS testing}\label{sec:adsTestingComplexity}

As we have seen, \acrshortpl{ads} can perform several tasks, in several environments. As such, there
are several relevant factors for testing them. It is not feasible to test all potential variations
of all potential environments in the real world, meaning that the \textit{test
    coverage}\footnote{See \Nref{sec:testCoverage}} typically will be low.

Some of the factors that complicate \acrshort{ads} operations are \begin{inparaenum}
    \item timing,
    \item sequence of events, and
    \item parameter settings such as the different speeds of various vehicles and other actors.
\end{inparaenum}

\citeauthor{adsComplexityIndex18} posit that \textit{the concept of complexity exists everywhere,
    but there is no agreement on one for driving situations}~\cite[1182]{adsComplexityIndex18}.
Therefore they introduce their own concept of \acrfull{dsc}, which serves to give a metric of a
the complexity of a given driving situation. Their \acrshort{dsc} is defined as the output of a
mathematical formula taking into account the perplexity and standard deviation of several
control variables $\mathcal{M}$ representing the surrounding vehicle's
behaviour~\cite[1182]{adsComplexityIndex18}. Their formula also takes into account the ratio of
\textit{V2X}-capable vehicles~\cite[1182]{adsComplexityIndex18}, i.e. the vehicles that are
connected and capable of communicating~\cite[1]{v2xTestingSurvey2019}.

% It is common to modularize the testing of \acrshort{ads} so that the individual modules can be
% tested in isolation. The modules of an \acrshort{ads} typically include a motion planner, a
% \acrfull{cv} system, and % LIDAR. This allows for testing the individual modules in a way that
% makes sense for their specific domains.

\subsection{ADS simulation}

Due to the complexity involved in testing \acrlongpl{ads} (\Cref{sec:adsTestingComplexity}),
simulators are typically used for this purpose~\cite{DeepScenario}. While the same points about not
being able to test \textit{all} possible scenarios do remain true for simulator based testing due to
the sheer number of factors, using a simulator allows for far greater testing at far lower cost due
to the minimal overhead of
\begin{inparaenum}
    \item generating,
    \item running, and
    \item evaluating the outcome of
\end{inparaenum}
test cases.

Furthermore, simulators allow for greater flexibility in determining the test scenarios due to not
being confined by the  physical world that is available to the scientist that wishes to perform the
testing. Using a simulator, a Europe-based scientist can test their \acrshort{ads} for North
American conditions, or vice-versa.

\subsection{The ADS simulator jungle}\label{sec:simulatorOverview}

Due to the appeal of running \acrshort{ads} simulation, several contenders exist
on the market.

\emph{Carla} is a widely used \acrshort{ads} simulator~\cite{Carla}. It is implemented
using the game engine UnrealEngine~\cite{unrealengine} and allows for running
test cases under various scenarios and collecting their results. Carla is fully
open source and is under active development. It has been applied in projects such as KITTI-Carla,
which generated a KITTI dataset using Carla~\cite{kittiCarla}.

\emph{LGSVL} is a deprecated simulator from LG~\cite{lgsvl}. It was used in projects such
as DeepScenario~\cite{DeepScenario}. It allowed for running various maps with various vehicles and
tracking their data. It was also capable of generating HD
maps \footnote{\url{https://github.com/lgsvl/simulator?tab=readme-ov-file\#introduction}}.
DeepScenario is a project similar to this, concerned with testing \acrlongpl{ads}. Further details
about it in are located in \Nref{sec:relatedWork}.

\emph{AirSim} is Microsoft's offering~\cite{airsim}. It has, like LGSVL,
been deprecated. It is also built using UnrealEngine. Unlike the other
simulators we have seen, this also focused on autonomous vehicles outside of
only cars, such as drones.


\subsection{Concepts of ADS simulation}\label{sec:adsSimConcepts}

\citeauthor{scenes} draw up an outline for the terms \textit{scene}, \textit{situation}, and
\textit{scenario}, that are all concepts widely used in \acrshort{ads} simulation testing.

\emph{scene} is a term that is used in different manners in various
articles~\cite[982]{scenes}, but \citeauthor{scenes} propose standardising the definition on
\textit{a scene describing a snapshot of the environment including the scenery and dynamic elements,
    as well as  as all actors’ and observers’ self-representations, and the relationships among those entities}~\cite[983]{scenes}.

\emph{situation} is, like \textit{scene}, employed in various fashions. \citeauthor{scenes}
give a background detailing its usage ranging from \textit{"the entirety of circumstances,
    which are to be considered by a robot for its selection of an appropriate behaviour pattern in a
    particular moment'}\footnote{The translation from German is borrowed from \citeauthor{scenes},
    \cite[984]{scenes}}, in  \citeauthor{scenarioTysk}~\cite[3]{scenarioTysk} to
\citeauthor{schmidtScenario} introducing a distinction between \textit{the true world} in a formal
sense, and that being the ground truth upon which a situation is
described~\cite[892]{schmidtScenario}.

\citeauthor{scenes} propose to standardise on the definition of a situation being \textit{
    the entirety of circumstances, which  are to be considered for the selection of an
    appropriate behaviour pattern at a particular point of time}~\cite[985]{scenes}.

\emph{scenario} refers to \textit{'the temporal development between several scenes in a sequence
    of scenes'}\cite[986]{scenes}. We note how the definition a a scenario utilises that of a scene.
Furthermore, \citeauthor{scenes} hold it to be the case that \textit{'every scenario starts with an
    initial scene. Actions \& events as well as goals \& values may be  specified to characterize
    this temporal development in a scenario'}~\cite[986]{scenes}, clarifying the distinction
between a scenario and a scene.

Lastly they posit that a scenario spans a certain amount of time, whereas a scene has no such
temporal aspect to it.


When running a simulation, we refer to the autonomous vehicle that is being
simulated as the \textit{ego vehicle}~\cite{egoDefinition}.

% introduce the concept of a \textit{scene}, which denotes 

% An alternate simulator is LG SVL~\cite{lgsvl}, but it has been deprecated since
% 2022 and as such it does not seem proper to build a new solution on top of it.
% The DeepScenario dataset~\cite{DeepScenario} utilises this simulator framework.
% \tanke{The LGSVL part can be removed - not relevant?}

\subsubsection{ADS scenario formats}\label{sec:adsScenarioFormats}

\emph{OpenSCENARIO} is a standard developed by the Association for Automation and
Measurement Systems (ASAM), which is dedicated to the description of dynamic
scenarios~\cite[651]{generatingOpenScenario}. Under this format, only the
\textit{dynamic} content of the scenario is recorded in the file. The static
content is kept in other formats such as OpenDRIVER and
OpenCRG~\cite[652]{generatingOpenScenario}. The simulator Carla (outlined in
\Cref{sec:simulatorOverview}) supports this
standard~\cite[652]{generatingOpenScenario}.

Another widely popular scenario format is
\emph{CommonRoad}~\cite[4941]{convOpenScenarioToCR}, first proposed in
\citedate{commonRoadOG}~\cite{commonRoadOG}. There are tools such as those
proposed by \citeauthor{convOpenScenarioToCR} that allows for converting
OpenSCENARIO scenarios to the CommonRoad
format~\cite[4941]{convOpenScenarioToCR}.

% TODO: Skrive om Carlas støtte for Python-scenarioer
% TODO: Skrive om Carla scenario runner her?

\section{Large language models (LLMs)}

\acrfullpl{llm} are transformer-based language models that typically contain several hundred billion
parameters and are trained on massive text data~\cite[4]{llmSurvey}.
Base language models, as the name implies, \textit{model language}. They are typically statistical
models and an example of \acrfull{ml}.

\subsection{Architecture}\label{sec:llmArch}

A \acrlong{llm} is a neural network trained on big data~\cite[3]{llmSurvey}. They expand on the
older statistical language models by training on more data. This gives rise to \textit{emerging
    abilities} such as in context learning~\cite[3]{llmSurvey} (\Nref{sec:emergentAbilities}). These
older statistical models are also neural networks, but they were impractical to train on large
amounts of data. It was not until the seminal paper \textsc{Attention is all you
    need}~\cite{attentionIsAllYouNeed} that a Google team headed by~\citeauthor{attentionIsAllYouNeed}
showed how neural networks can be trained in parallel using their new \textit{attention} mechanism.
This allowed for using amounts of data that was not technologically practical up until that point,
opening the door for later advancements such as
ChatGPT~\cite[9]{llmSurvey}

% \todo{Should elaborate further?}
\citeauthor{jm} describe how \acrshortpl{llm} rely on \textit{pretraining}.

\subsubsection{The importance of training data}

As a consequence of \acrshortpl{llm} being statistical models of a certain input
data~\cite[1]{llmSurvey}, what data the model is trained on is of great
importance for the capabilities of the model~\cite[6]{llmSurvey}.
\citeauthor{llmSurvey} give an overview of various \acrshortpl{llm} and what
kinds of corpora\footnote{A corpus (pl. corpora) refers to a document
    collection.} they have been trained on~\cite[11-14]{llmSurvey}.

The training data will provide the model with its base understanding of the
world, and as such it will dictate \begin{inparaenum}
    \item what it `knows', and
    \item how we should interact with it.
\end{inparaenum}
E.g., if we want to solve problems related to software code, we should employ a
model that has been \textit{trained} on software code related topics so that the
probability of it predicting correct tokens will be higher. If it has not seen
any code during its training it would not have any base `knowledge' for solving
our problem, and its output would be bad. The \acrshort{llm} would however have
no way of knowing if its output would be right or wrong, and we could say that
it would have \textit{hallucinated}.
See \Nref{sec:llmProblems} for further information
about hallucination.


\subsection{Emergent abilities}\label{sec:emergentAbilities}

\citeauthor{emergentabilitiesLLM} outline how \textit{emergent abilities} appear
when scaling up language models~\cite[1]{emergentabilitiesLLM}. They define
\textit{emergent ability} to refer to abilities that are not present in smaller
models, but present in the larger ones\cite[1]{emergentabilitiesLLM}, building
on physicist~\citeauthor{anderson1972more} stating that \textit{Emergence is
    when quantitative changes in a system result in qualitative changes in
    behaviour.}~\cite[2]{emergentabilitiesLLM}.

Furthermore, they discuss how \textit{few-shot prompting} typically can achieve
far superior results for harvesting \acrshort{llm} emergent abilities, whereas
one-shot prompting can perform worse than randomized
results~\cite[3-4]{emergentabilitiesLLM}.

They continue outlining several approaches for achieving augmented prompting
strategies, underlining how \begin{inparaenum}
    \item multi-step reasoning
    \item instruction following
    \item program execution,
    and
    \item model calibration
\end{inparaenum}
all serve as possible ways of increasing \acrshort{llm} performance~\cite[5]{emergentabilitiesLLM}.

% \tanke{Burde bygge på forrige seksjon om "små" language models}

\subsection{Intelligence in LLMs}\label{sec:llmIntelligence}

There are three theories on machine intelligence, each serving to
explain how they `\textit{think}': \begin{inparaenum}
    \item stochastic parrot
    \item Sapir-Whorf hypothesis,
    and
    \item conceptual blending.
\end{inparaenum}

\subsubsection{Stochastic parrot}\label{sec:llmParrot}

\citeauthor{parrot} outline how \acrshortpl{llm} can \textit{fool} humans as they
are trained on ever larger amounts of parameters and data, appearing to be in possession of an
intelligence~\cite[610-611]{parrot}.

This anticipates the phenomenon of hallucination (\Cref{sec:llmHallucination}).

\subsubsection{Sapir-Whorf hypothesis}

The Sapir-Whorf hypothesis posits that  \textit{The structure of anyone’s native
    language strongly influences or fully determines the world-view he will acquire
    as he learns the language.}~\cite[128]{sapirWhorf}.

We note how this maps to our \acrshortpl{llm}, indicating that they will only ever
be able to `know' the data on which they have come into contact with.

Or: \emph{Language} defines the possible room for \emph{thought}.


\subsubsection{Conceptual blending}
%Relevant?

Conceptual blending is a theory on intelligence. It refers to the basic mental
operation that leads to new meaning or insight that occurs when one identifies
a match between to input mental spaces, to project selectively from those inputs
into a new `blended' mental space~\cite[57-58]{conceptBlending}.

This phenomenon explains how we are able to imagine phenomena that logically
should not exist such as \textit{land yacht} (\Nref{fig:landYacht})

\begin{figure}[h]
    \centering
    \includegraphics[width=0.8\textwidth]{figures/landYacht.png}
    \caption[Land yacht conceptual blend]{The conceptual blend of a \textit{land
            yacht}\footnotemark}\label{fig:landYacht}
\end{figure}

\footnotetext{Diagram borrowed from \citeauthor{conceptBlending},~\cite[67]{conceptBlending}.}

We note how this is how \acrshortpl{llm} operate when processing vectorized
linguistic data.
% Conceptual blending is a theory on both machine and human intelligence. 


\subsection{Utilising LLMs - Prompt engineering}\label{sec:llmUtilization}


A typical way of interacting with \acrshortpl{llm} is \textit{prompting}~\cite[44]{llmSurvey}. You
prompt the model to solve various tasks. As we saw in \Nref{sec:emergentAbilities}, the level of
performance you are able to extract from your \acrlong{llm} can depend a great deal on how you
interact with it. The process of manually creating a suitable prompt is called \emph{prompt
    engineering}~\cite[44]{llmSurvey}.~\citeauthor{llmSurvey} outline three principal prompting
approaches:

\emph{\acrfull{icl}} is a representative prompting method that formulates the task
description and/or demonstrations in natural language text~\cite[44]{llmSurvey}. It is based on
\textit{tuning-free prompting} and it, as the name implies, never tunes the parameters of the
\acrshort{llm}~\cite[15]{promptingSurvey}. One the one hand, this allows for efficiency, but on the
other hand, heavy engineering is typically required to achieve high accuracy, meaning you must
provide the \acrshort{llm} with several answered prompts~\cite[16]{promptingSurvey}. In layman's
terms, \acrshort{icl} entails including examples of the process you want the model to perform when
prompting it.

\emph{\acrfull{cot} prompting} is proposed to enhance \acrlong{icl} by involving a
series of intermediate reasoning steps in prompts~\cite[44, 52]{llmSurvey}. The basic concept of
\acrshort{cot} prompting, is including an actual \acrlong{cot} inside the prompt that shows the way
form the input to the output~\cite[52]{llmSurvey}.~\citeauthor{llmSurvey} note that the same effect
can be achieved by including simple instructions like `\textit{Let's think step by step}' and other
similar `magic prompts' in the prompt to the \acrshort{llm}, making \acrshort{cot} prompting easy to
use~\cite[52]{llmSurvey}.

\emph{Planning} is proposed for solving complex tasks, which first breaks them down into smaller
sub-tasks and then generates a plan of action to solve the sub-tasks one by
one~\cite[44, 54]{llmSurvey}. The plans are being generated by the \acrshort{llm} itself upon
prompting it, and there is a distinction between text-based and code-based approaches. Text-based
approaches utilise natural language, whereas code-based approaches utilise executable computer code~\cite[54-55]{llmSurvey}.


\subsection{General challenges with LLMs}\label{sec:llmProblems}

We have seen that \acrshortpl{llm} demonstrate promising abilities (\Nref{sec:emergentAbilities}) But they have nevertheless certain issues attached to them that we need to be aware of.

\subsubsection{Hallucination}\label{sec:llmHallucination}

As we saw in \Cref{sec:llmParrot}, \acrshortpl{llm} are prone to
\textit{bullshitting}. They have no intuition of, or concern with \textit{the
    truth}. They only ever yield whatever response is the most probable under their
\textsc{beam search} algorithm being applied on their training data.

\subsubsection{Environmental concerns}

A University of Rhode Island study on the environmental impact of \acrshortpl{llm} have shown that
they require wast amount of energy and water~\cite{hungryLlm}. They also found that the different
\acrshortpl{llm} may differ greatly in their energy consumption, highlighting that that certain
\acrshortpl{llm} may consume more than \num{70} times more energy than others~\cite{hungryLlm}.

Another study by \citeauthor{llmCarbon} focusing specifically on \textit{carbon emissions} did
however find that these emissions significantly lower for \acrshortpl{llm} than humans for specific
tasks such as text and image generation, ranging from \num{130} to \num{2900} times less Co2 emitted
depending on the task~\cite[1]{llmCarbon}.

\citeauthor{thirstyLlm} surveyed the water consumption of \acrshortpl{llm}, finding that training the
\acrshort{llm} \textsc{GPT-3} could evaporate as much as \num{700000} litres of clean
freshwater~\cite[1]{thirstyLlm}. Furthermore they review the trends of current AI adoption and
project that the water consumption of AI could reach levels as high as \num{4.2} - \num{6.6} billion
cubic metres by \num{2027}, which is comparable to \num{4} - \num{6} Denmarks, or half of the United
Kingdom~\cite[1]{thirstyLlm}. Recent research indicates that \textit{serving} \acrshortpl{llm}
currently account for more emissions than training them~\cite[37]{sustainableLlmServing}.

Efforts to achieve greener \acrshortpl{llm} have been proposed by \citeauthor{sproutGreenLlm}, while
recognizing the trade-off between ecological sustainability and high-quality
outputs~\cite[21799]{sproutGreenLlm}.

% TODO: Include or not?
% \subsubsection{Cognitive atrophy}
% https://arxiv.org/abs/2506.08872 

% TODO: Include or not?
% \subsubsection{LLM collapse}
% https://machinelearning.apple.com/research/illusion-of-thinking 


\subsection{The different kinds of LLMs}\label{sec:llmJungle}

There are several available \acrshortpl{llm}, some of which are open source, and some proprietary.
Open source \acrshortpl{llm} afford greater insight into their composition and underlying training
data, whereas proprietary models appear more like black boxes. Some popular model families include
the GPTs, Gemini, Llama, Claude, Mistral, and DeepSeek.

The \acrshortpl{llm} differ primarily in their \begin{inparaenum}
    \item parameters, and
    \item training data.
\end{inparaenum}
As we saw in \Cref{sec:llmArch}, all typical \acrshortpl{llm} utilise a transformer-based neural
network. But due to their various different properties, different models can behave differently for
different tasks regardless of their similar architecture.

What they all share is their ability to perform \textit{inference}, meaning that they predict output
tokens given some input tokens (see \Cref{sec:llmParrot}).

\subsection{Existing LLM applications for ADS}\label{sec:llmsForAds}


\citeauthor{LLM4AD} give a broad overview of some of the ways \acrshortpl{llm} have been applied for
\acrshortpl{ads}, highlighting some of the opportunities and potential weaknesses of \acrshort{llm}
applications for \acrshort{ads} purposes. One of the ways \acrshortpl{llm} can be applied, is for
adjusting the driving mode, or aiding in the decision-making
process~\cite[1]{LLM4AD}.~\citeauthor{driveAsYouSpeak} delve further into these aspects in their
other work \citetitle{driveAsYouSpeak}, providing a framework for integrating \acrlong{llm}'s
\begin{inparaenum}
    \item natural language capabilities,
    \item contextual understanding,
    \item specialized tool usage,
    \item synergizing reasoning, and
    \item acting with various modules of the \acrshort{ads}
\end{inparaenum}~\cite[1]{driveAsYouSpeak}.
\chapter{Problem description}\label{sec:problemDescription}

\epigraph{A problem well stated is a problem half solved.}{\textit{Charles F. Kettering}}

\section{Cost and bias}

Traditional techniques for obtaining \acrshort{ads} scenarios rely on high skilled manual labour.
This incurs a significant cost, and is a major limitation in obtaining a large number of good
scenarios, free from the bias of the author
% TODO: Bias er ikke relevnt her??


\section{Impossible to test all scenarios}
Furthermore, even if we were to imagine a world in which we had infinite \begin{inparaenum}
  \item time and
  \item money
\end{inparaenum}, we would not be able to successfully account for every possible scenario. This is
a reality we need to deal with. One possible measure of remedying with this, could be to
\textit{decrease} the driveability of our existing scenarios. Decreasing the driveability is not the
same as suddenly having access to the infinite set of possible scenarios, but it is reasonable to
infer that begin able to \textit{test} the \acrshort{ads} (in a simulator) on these enhanced
low-driveability scenarios will leave it better fit for encountering other low-driveability
scenarios in the wild during operation.

\section{Edge cases}

Edge cases can be a major issue for \acrshort{ads} adoptation. The \textit{tail problem} as it is
known in the \acrshort{ml} field posits that \acrshort{ml} tasks are faced with a long tail of
unseen cases. We can map these unseen cases, to our unseen \acrshort{ads} scenarios. Because of
this, an \acrshort{ads} can be at risk of encountering an unseen edge case scenario during
operation -- something for which it might never have been tested.
% TODO: Burde referere ting om tail problem? 
Arguing that the \acrshort{ads} would probably crash simply due to it finding itself in an unseen
scenario is not logical. But it is important to keep in mind that the end we are pursing in the
broader adaption of \acrfullpl{ads}, is increased saftey and efficiency on our roads. Not
sufficiently testing the \acrshort{ads} before deploying it would not serve our goal of increasing
road safety -- it would be a gamble with human lives.
\chapter{Literature review}

TODO: Write literature review

Can move some things from related work such as LLM4AD?

\section{Graz University of Technology survey on \acrshort{llm} applications for
\acrlongpl{ads}} % TODO: Det er OK å referere til denne surveyen som dette?

\citeauthor{surveyLLMScenarioBasedTesting} give an extensive overview of some of the various ways
that \acrshortpl{llm} have been applied to scenario based testing of \acrlongpl{ads}.
The authors classify the various research efforts based on \begin{inparaenum}
    \item how they have employed the \acrshort{llm}, and
    \item to what end
\end{inparaenum}~\cite{surveyLLMScenarioBasedTesting}.
Their survey is continually updated, the last update having been made 2 months before the time of
writing\footnote{I.e. as of September 17th 2025, the last update to their
    \href{https://github.com/ftgTUGraz/LLM4ADSTest}{Github repo} was on July 23rd, 2025. The paper on
    Arxiv was last updated May 22nd 2025.}. This entails a certain overlap with some of the works we
review in \Nref{sec:relatedWork}.
% Not deterred by this, let us look at how they classify the works:

Not deterred by this, let us delve into the survey:
They start by highlighting the trend between the number of \acrshort{llm} surveys, and
\acrshort{ads} surveys -- while the trend was increasing from 2020-23, there was an explision in
\num{2024}, with about \num{200} works concering applying \acrshortpl{llm} for \acrlong{ads}
purposes being published~\cite[p. 1, figure (b)]{surveyLLMScenarioBasedTesting}. Furthermore, the
number of \acrshort{ads} studies has remained steady over the last \num{4}  years, wheras the number
of \acrshort{llm} studies has exploded in popularity~\cite[p. 1, figure
    (a)]{surveyLLMScenarioBasedTesting}. This indicates that a significant amount of the scientific
effort around \acrshortpl{ads} the last year, has been concerned with utilising \acrshortpl{llm}.
% TODO: Er det OK at jeg gjør utledninger som dette? (uten noe referanse)

\subsection{Meta survey review}

The article summarizes the field, pulling together various surveys of the
related subfields. Those being \begin{inparaenum}
    \item \acrshort{llm} surveys,
    \item surveys of scenario-based testing,
    \item general cases of \acrshortpl{llm} for \acrshortpl{ads}, and finally
    \item a broader review of surveys of \acrshortpl{llm} being applied for
    \textit{miscellaneous domains}
    \end{inparaenum},
for each highlighting their specialized
foci~\cite[2]{surveyLLMScenarioBasedTesting}.
% TODO: Plural of "focus" is "foci", yes?

\subsection{The categories of ways of applying \acrshortpl{llm} for \acrshort{ads} testing}

The authors posit that there are \num{0} major categories of works of
\acrshortpl{llm} being applied to \acrlongpl{ads}. They are.

\subsection{The \num{5} key challenges when applying \acrshortpl{llm} for
\acrshort{ads} testing}

Furthermore


\part{The project}
% \chapter{Planning the project} 

\chapter{Related work}\label{sec:relatedWork}

\epigraph{Learn from the mistakes of others. You can't live long enough to make them all yourself.}{Eleanor Roosevelt}

This chapter surveys several related works. It contains a selection of works that are typically
related to applying \acrshortpl{llm} specifically or \acrshort{ml} more generally to \acrshort{ads}
simulator scenarios.

\section{DeepScenario}\label{sec:deepScenario}

DeepScenario is both a dataset and a toolset aimed at \acrlong{ads} testing~\cite{DeepScenario}. The
principal value proposition of this work lies in recognizing the fact that \begin{inparaenum}
  \item there are an infinite number of possible driving scenarios, and
  \item generating critical driving scenarios is very costly with regard to time costs and
  computational resources\end{inparaenum}~\cite[52]{DeepScenario}. The authors therefore propose
an open driving scenario of more than \num{30000} driving scenarios focusing on \acrshort{ads}
testing~\cite[52]{DeepScenario}. The project utilises traditional machine learning
methodologies, having been performed prior to the broad adaptation of \acrshortpl{llm}.

Its scenarios are intended for the simulator SVL by LG (\Cref{sec:simulatorOverview}).

\section{RTCM}

RTCM is a \acrshort{ads} testing framework that allows the user to utilise natural language for
synthesizing test cases. The authors propose a domain-specific language --- called RTCM, after
\textsc{Restricted Test Case Modelling} --- for specifying test cases. It is based on natural language
and composed of \begin{inparaenum}
  \item an easy-to-use template,
  \item a set of restriction rules, and
  \item keywords \end{inparaenum}~\cite[397]{RTCM}.  Furthermore, they also propose a tool to
take this RTCM source code as input and generating either \begin{inparaenum}
  \item manual, or
  \item automatically \end{inparaenum} executable test cases~\cite[397]{RTCM}. The proposed tools
were evaluated in experiments with industry partners, successfully generating executable test
cases~\cite[397]{RTCM}.

\section{DeepCollision}

\citeauthor{deepCollision} utilise \acrfull{rl} for \acrshort{ads} testing, with the goal of getting
the \acrshort{ads} to \textit{collide}. They used \textit{collision probability} for the loss
function of the \acrlong{rl} algorithm~\cite[384]{deepCollision}. Their experiments included
training 4 DeepCollision models, then using \begin{inparaenum}
  \item random, and
  \item greedy
\end{inparaenum} models for generating a baseline to compare their models with. The results showed
that DeepCollision demonstrated significantly better effectiveness in obtaining collisions than the
baselines. While not specifically focused on \textit{testing}, we recognize that their work is thematically
similar to our envisioned project.

\section{AutoSceneGen}\label{sec:autoSceneGen}

AutoSceneGen is a framework for \acrshort{ads} testing using \acrshortpl{llm},
focusing on the motion planning of \acrlong{ads}~\cite[14539]{autoSceneGen}.
\citeauthor{autoSceneGen} highlights how \acrshortpl{llm} provide opportunities
for efficiently evaluating \acrshort{ads} in a cost-effective
manner~\cite[14539-14540]{autoSceneGen}. They generate a substantial set of synthetic scenarios and
experiment with using \begin{inparaenum}
  \item only synthetic data,
  \item only real-world data, and
  \item a combination of the \num{2} \end{inparaenum} as training data. They find that motion
planners trained with their synthetic data significantly outperforms those trained solely on
real-world data~\cite[14539]{autoSceneGen}.

\section{LLM4AD}

LLM4AD is a paper that gives a broad overview of \acrshortpl{llm} for \acrlong{ads}. It touches on
several of the various \acrshort{ads} applications where \acrshortpl{llm} are relevant such as
\begin{inparaenum}
  \item language interaction,
  \item contextual understanding,
  \item zero-shot and few shot planning allowing \acrshortpl{llm} to perform tasks they weren't trained
  on, helping with handling edge cases
  \item continuous learning and personalization, and finally
  \item interpretability and trust \end{inparaenum}~\cite[2]{LLM4AD}. Furthermore, the authors
also propose a comprehensive benchmark for evaluating the instruction-following abilities of an
\acrshort{llm} based system in \acrshort{ads} simulation~\cite[1]{LLM4AD}.

\section{LLM-Driven testing of \acrshort{ads}}

\citeauthor{LLMDrivenTestingADS24} worked on using \acrshortpl{llm} to for automated test generation
based on free-form textual descriptions in the area of automotive~\cite[173]{LLMDrivenTestingADS24}.
They propose a prototype for this purpose and evaluate their proposal for \acrshort{ads} driving
feature scenarios in Carla. They used the \acrshortpl{llm} GPT-4 and Llama3, finding GPT-4 to
outperform Llama3 for the stated purpose. Their findings include this \acrshort{llm}-powered test
methodology to be more than \num{10} times faster than traditional methodologies while reducing
cognitive load~\cite[173]{LLMDrivenTestingADS24}.
% TODO: Cognitive load -> brain atrophy (sec:llMproblems)

\section{Requirements All You Need?}

\citeauthor{requirementsAllYouNeed} provide an overview of \acrshortpl{llm} for \acrshort{ads} in
their recent preprint~\citetitle{requirementsAllYouNeed}\footnote{This was submitted to Arxiv on
  2025-05-19.}, focusing on \acrshort{llm}'s abilities for translating abstract requirements extracted
from automotive standards and documents into configuration for Carla (\Cref{sec:simulatorOverview})
simulations~\cite{requirementsAllYouNeed}. Their experiments include employing the
\textit{autonomous emergency braking} system and the sensors of the \acrshort{ads}. Furthermore, they
split the requirements into \num{3} categories: \begin{inparaenum}
  \item vehicle descriptions,
  \item test case pre-conditions, and
  \item test case post-conditions (\Nref{sec:testingConditions})
\end{inparaenum}~\cite{requirementsAllYouNeed}. The preconditions they used included
\begin{inparaenum}
  \item agent placement,
  \item desired agent behaviour, and
  \item weather conditions amongst others\end{inparaenum}, whereas their postconditions reflected
the desired outcomes of the tests, primarily related to the vehicle's
telemetry~\cite{requirementsAllYouNeed}.

\section{Language Conditioned Traffic Generation}

\citeauthor{languageconditionedtrafficgeneration} look into using \acrshortpl{llm} to generate
specific traffic scenarios. They identify the importance of being able to use simulators to test
\acrshortpl{ads}, and highlight how test scenarios are expensieve to
obtain~\cite[1]{languageconditionedtrafficgeneration}. To this end, they propose a tool --
\textsc{LTCGen} which employs the strengths of \acrshortpl{llm} to match a natural language query
with a fitting underlying map\footnote{Map as in a \textit{world} in which a scenario can take
  place.}, and populates this with a \begin{inparaenum}
  \item initial traffic distribiution, and
  \item the dynamics of all the vehicles involved in the scene.
\end{inparaenum}
Something to note is that they generate their scenarios, without initially taking the \textit{ego
  vehicle} into account. The ego vehicle of the scene is simply determined as the vehicle that is
in the \textit{center} of the first
\textit{frame}~\cite[3]{languageconditionedtrafficgeneration}.

\section{Scenario engineer GPT}

\citeauthor{seGpt} outline a framework for utilising the \acrshort{llm}-backed ChatGPT in order to
generate scenarios. They propose SeGPT -- a scenario generation framework that they found to yield
\textit{significant progress in the domain of scenario generation}~\cite[4422]{seGpt}. They posit
that their prompt engineering ensures that the generated scenarios are authentically diverse and
challenging~\cite[4423]{seGpt}. The focus is primarily on \textit{trajectory
  scenarios}~\cite[4422-4423]{seGpt}.

% TODO: Dette avsittet virker litt malplassert - det virker mer som discussion enn RW
Note how they explicitly mention scenario \textit{generation}. Our approach for this project has a
different angle, with the focus being on modifying \textit{existing} scenarios. More on this in
\Nref{sec:solutionProposal}. The difference between generating a `brand new' scenario with a model
trained on exisiting scenarios, and modifying an existing scenario seems like a matter of
granularity. These are very similar concepts, only that the enhanced scenario will have more common
DNA whereas the other `new' scenario will consist of a broader range of DNA from its various
underlying scenario corpora.

\section{LLM driven scenario generation}

\citeauthor{LLMScenarioChang24} also look into using \acrlongpl{llm} to generate \acrshort{ads}
scenarios. They recognize several of the challenges we outline in \Cref{sec:problemDescription}. In
their \citeyear{LLMScenarioChang24} paper, they propose \textsc{LLMScenario}, which is an
\acrshort{llm}-backed framework for both \begin{inparaenum}
  \item scenario generation, and
  \item  evaluation feedback tuning
\end{inparaenum}~\cite[6581]{LLMScenarioChang24}.

They analyze scenarios in order to provide the \acrshort{llm} with a minimum baseline scenario
description, and propose score functions based on both \begin{inparaenum}
  \item reality and
  \item rarity.
\end{inparaenum} Their prompting is based on \acrfull{cot} and a posteriori emperical experience.
Lastly, they tested several \acrlongpl{llm} for their experiments. Their results were positive,
indicating effectiveness for scenaro engineering in Industry 5.0~\cite[6581]{LLMScenarioChang24}.
% TODO: Burde finne ut av hva in nomine christi "industry 5.0" er så jeg ikke risikerer å skrive noe
% som ikke gir mening.

\section{Chat2Scenario}

\citeauthor{chat2Scenario} propose a method for utilising \acrshortpl{llm} to retrieve
\acrshort{ads} scenarios given a natural language query. Their framework synthesizes scenarios from
naturalistic\footnote{Their term. The intended meaning of \textit{naturalistic} is not all clar to
  me.} driving datasets, based on observation real world human driving~\cite[55]{chat2Scenario}, that
it then uses as a database for retrieveing the scenario that best matches the user's natural
language query. Furthermore, they employ traditional techniques for asserting the relevance of the
retrieved scenarios, allowing the user to specify a set of \textit{criticality metrics}, of which a
certain threshold must be reached amongst the scenarios that are initalliy retried by the
\acrshort{llm}, pruning false positives. As a measure to increase the usability of their framework,
they also provide a webapp with an intuitive \acrshort{gui} for both \begin{inparaenum}
  \item operating the tool, and
  \item visualizing the scenarios \end{inparaenum}~\cite[560]{chat2Scenario}.

In order to allow the \acrshort{llm} to determine whether a scenario is relevant under the
provided query, they put forward a method for classifying the various scenarios using traditional
\acrshort{ml} techniques. This classification focuses primarily on highway scenarios and the
activities of other actors in relation to the ego vehicle~\cite[561-562]{chat2Scenario}.

\subsubsection*{Prompt engineering}

% TODO: Legge inn kryssreferanse til background om prompt engineering? 

The project's prompts are `informed' by the \num{6} \acrlong{oai} guidelines from their prompt
engineering guide\footnote{\url{https://platform.openai.com/docs/guides/prompt-engineering} (URL
  from the paper.)}, ending up with a structured prompt of \num{5} segments. These segments serve to
guide the \acrshort{llm}, delineating its role as an `advanced \acrshort{ai} tool for scenario
analysis, specifically tasked with interpreting driving scenario following a pre-established
classification model'~\cite[562]{chat2Scenario}. They then input the user-provided description of
the scenario they wish to retrieve. Following this, a third segment declares the format for the
\acrshort{llm} response, followed by a prime example of \acrlong{icl}, demonstrating what a
satisfactory fulfillment of the desired format could look like. Lastly they instruct the
\acrshort{llm} to \textit{Remember to analyze carefully and provide the classification as per the
  structure given above}~\cite[563]{chat2Scenario}.
% TODO: Ref siste punkt om "husk å gjøre det riktig" -> kan skrive om dette fenomenet i background
% og så referere tilbake til det.

\section{Predicting driving comfort in autonomous vehicles using road information and multi- head
  attention models}

The \citeyear{Chen2025} article of \citeauthor{Chen2025}~\cite{Chen2025}, delves into the various
aspects related to predicting driving comfort in autonomous vehicles based on \begin{inparaenum}
  \item available road information, and
  \item multi-head attention models.
\end{inparaenum}
Their principal focus is on driving \emph{comfort}. To this end, they evaluate \acrshortpl{ads} in
light of the \textbf{jerk} metric in various situations. Furthermore, they highlight how a high
complexity in the scenarios can increase the probability of emergency breaking occurring, which is
naturally antithetical to comfort for the \acrshort{ads} operator and their passengers.

In order to measure this comfort, they rely metrics calculated from datapoints from the
\acrshort{ads} system -- jerk and acceleration. This, they use in conjunction with manual human
driving evaluation scores, to compose a new metric, the `driving comfort evaluation score'
(DCES)~\cite[10]{Chen2025}.

Moreover, they use this information to propose a model -- the \acrfull{adcp} model -- for
\emph{predicting} driving comfort from road information~\cite[2]{Chen2025}.
\chapter{\hefe~implementation}\label{chp:solutionProposal}

\epigraph{The only difference between a problem and a solution is that people understand the solution.}{Charles F. Kettering}

Recall that have seen in the \Nref{chp:introduction} and \Nref{chp:background}
that there are several complexities involved with \acrshort{ads} testing, such
that we typically use simulator-based testing to aim at verifying the safety of
the \acrshort{ads} before it gets deployed to the real world. But then we saw
further that simulator-based testing can lead to a false sense of safety due to
the \acrshort{ads} passing all our test scenarios, and then there can be a hole
in what they test for, so that the \acrshort{ads} may actually have
undiscovered faults that we haven't caught on to.

Futhermore, we saw that \acrshortpl{llm} have capabilities for modifying textual
data (such as e.g. scenario definitions) by natural language prompt engineering.
In \Nref{chp:relatedWorkAndLitReview} we surveyed several projects that aim to
enhance scenario-based testing of \acrshortpl{ads}, and we saw that some of them
employed \acrshortpl{llm}. But no extant research has specifically used
\acrshortpl{llm} to \emph{decrease driveability} of \acrshortpl{ads}
scenarios\footnote{The most similar work, to the author's knowledge,
is~\cite{yao2025agentsllm} -- \citetitle{yao2025agentsllm}.}. 

This research gap is addressed by this thesis -- we propose a novel
\acrshort{llm}-powered methodology for decreasing the driveability of
\acrshort{ads} scenarios so that when the \acrshort{ads} is tested with the less
driveable test scenarios, we can either \begin{inparaenum}
    \item trigger it to fail and analyse what went wrong, or 
    \item be more confident in the \acrshort{ads} being able to operate in
    complex scenarios due to passing them.
\end{inparaenum}

To this end, we propose \hefe\footnote{`'\acrshortpl{llm} for decreased driveability'.}~-- a tool for
\begin{inparaenum}
    \item running a base \acrshort{ads} test case,
    \item enhancing the test case using \acrshortpl{llm},
    \item running the enhanced test case, and
    \item comparing the results of the two runs.
\end{inparaenum}
We summarize the motivation behind using the \hefe~tool in the following user story:

\begin{enumerate}
    \item I have a set of \acrshort{ads} test scenarios. I provide this
            set to \hefe. It will run the entire set, and generate a
            baseline of my \acrshort{ads} performance.

    \item \hefe~will then improve my test scenarios using
        \acrshortpl{llm} to make them less driveable,  and run the
        enhanced versions.

    \item Lastly \hefe~will report how the results differ from running
        the base and enhanced version of a test case.

    \item This will give me insight into my \acrshort{ads} by reviewing
        what scenarios it failed to complete Then I can look into the
        cause of the error state and uncover underlying faults in the
        \acrshort{ads} that i did not know about beforehand. If the
        \acrshort{ads} \emph{is} able to complete the less driveable
        scenarios, I can be more confident in it and be more assured
            that it will work properly during real-world operation.
\end{enumerate}

The tool follows a natural pipeline structure. We have some base test scenarios that
need to be run in order to get a baseline for the results, we then have to
enhance these, and run the improved versions and compare them to their original
versions.
Experimenting with this tool will allow us to learn the extent to which
\acrshortpl{llm} can be applied for decreasing  driveability in \acrlong{ads}
test scenarios, satisfying the \Nref{sec:problemDescription}. The tool is
implemented in an \acrshort{llm}- and scenario-agnostic fashion so that it is
scalable to other combinations of \acrshort{llm} and scenario formats than those
experimented with in this specific work to verify the feasibility of the tool.

\begin{figure}[htb]
    \centering
    \includegraphics[width=0.95\textwidth]{experiment-material/accident-pics/base/underway.png}
    \caption{A screenshot from executing a Carla scenario.}\label{fig:carlaScenarioExample}
\end{figure}

\Cref{fig:carlaScenarioExample} renders an example of how it can appear when exeucitng a test
scenario on the Carla simulator. Runs like this will later ber presented in \Nref{chp:results} and
then analysed in \Nref{chp:discussion} to evaluate the value of the \hefe~tool.

\section{\hefe~architectural overview}

In order to decrease the driveability of the \acrshort{ads} scenarios, we need \num{3}
separate components: \begin{inparaenum}
    \item something to handle the \acrshort{llm} interfacing, 
    \item something that can integrate with the \acrshort{ads} simulator, and
    finally 
    \item some kind of human-facing interface to administrate the process. 
\end{inparaenum}
By compartmentalizing what component has responsibility for what task, we reduce
complexity and increase the possibility of repurposing the modules for other
possible tasks in the future. UNIX philosophy!
These components and their relationship is rendered in \Cref{fig:hefeArch}.

For the sake of making their roles more clear, we christen the components as
follows: \begin{inparaenum}
    \item Odin will be the module for interfacing with \acrshortpl{llm} and
    performing the enhancement.
    \item Thor will take \acrshort{ads} scenarios, run them on a simulator and
    report the results. Finally,
    \item Loki will interface with the human and start the process by
    determining what \acrshort{llm} is to be used with what prompts with what scenarios.
\end{inparaenum}

\begin{figure}[h]
    \centering
    \includegraphics[width=\textwidth]{figures/d2-pdf/hefe.pdf}
    \caption{\hefe~pipeline architecture}\label{fig:hefeArch}
\end{figure}

All the prompts exists as a part of the Odin module. Loki will request available
prompts from Odin on behalf of the user, and the user will select which one they
want to use. Similarly, all the scenarios exist within the Thor module and are
handled by it. As for the prompt, the user reqquests a list of available
scenarios from the Thor module and makes a decision on which they want to use. 



\subsection{Implementation language}

The programming language \textsc{Python} is widely used for \acrshort{ads} simulation. It is a high
level language, allowing the user great flexibility and developer experience. For this reason, I
will implement \hefe~using Python.
Python can be optimized using \acrfull{jit} compilers such as Numba~\cite{numba}, which can speed up
our execution times. Libraries such as Joblib provide Python with plug-and-play
meomization, which will allow us to re-use values that have already been
computed, saving time and energy.


\subsection{Overview of the components of the \hefe~pipeline}\label{sec:hefeComponentOverview}

The pipeline architecture is visualised in \Nref{fig:hefeArch}. Here we
present the major components and their responsibilities.


\subsection*{Test case enhancement}

\subsubsection{Test case repositories}

We have seen in \Nref{sec:relatedWork} that there are existing repositories of
% TODO: Make more specific reference to what part of Related Work
\acrshort{ads} test cases. These will provide us with \begin{inparaenum}
    \item a baseline,
    and
    \item data onto which we can apply our \acrshort{llm} enhancements.
\end{inparaenum}

We do naturally have to walk before we can run. For this reason, the project will initially be
tested on simple test scenarios provided by people behind the Carla simulator. When we have verified
that the project is sufficiently working for its stated purpose, we can scale up the activities to
other datasets. Several are presented in \Nref{sec:relatedWork}. The concept of applying
\acrshortpl{llm} to \acrshort{ads} scenarios is quite universal in nature and is eligible for
application for virtually \emph{all} datasets.

\subsubsection{LLM enhancement}\label{sec:llmEnhancement}
% TODO: Dette må spisses mot driveability

The base test cases will individually be enhanced by prompting the
\acrshort{llm}. We will experiment with several \acrshortpl{llm}.

Note that we will not employ any traditional \acrfull{nlp} techniques related to e.g. tokenization
or input processing -- we will leave this up to the internal mechanisms of the \acrshortpl{llm}.

For performing the actual improvement, it is essential that we \begin{inparaenum}
    \item test several \acrshort{llm},
    \item give clear prompts
    % \info{I'm inclined to find some fitting prompts through trial and error, as such a I do not wish to describe them in detail at this time.}
    and
    \item verify that the returned test case adheres to the strictly necessary
    syntax rules. This last point is important due to our knowledge of
    \acrshortpl{llm} hallucinating (see \Nref{sec:llmProblems}).
    % TODO: More specific reference to Halucination intead of all LLm problems?
\end{inparaenum}

In order to facilitate testing various \acrlongpl{llm}, we should employ
\acrshort{llm} agnostic software as a translation layer. This will allow us to
write code for a common interface and test several \acrshortpl{llm} that may all
have different internal \acrfullpl{api} without having to modify our test code
for specific \acrshortpl{api}. This \begin{inparaenum}
    \item saves time
    and
    \item makes for more even test conditions \end{inparaenum}. Some pieces of software providing
this type of functionality include
\textsc{aisuite}\footnote{\url{https://github.com/andrewyng/aisuite}}, RamaLama from
RedHat\footnote{\url{https://github.com/containers/ramalama}}, and the MIT licensed
Ollama\footnote{\url{https://github.com/ollama/ollama}}, both supporting a plethora of
\acrlongpl{llm}.

% TODO: Burde fjerne disse random Github-prosjektene som ikke blir brukt i den faktiske implementasjonen?
\textsc{guidance}\footnote{\url{https://github.com/guidance-ai/guidance}} is a
framework for limiting the room in which \acrshortpl{llm} may operate, which
might be useful if we run into issues with excessive hallucination.


\subsubsection{Enhanced test case validation}

We must expect the \acrshort{llm} to hallucinate to some extent (\Cref{sec:llmHallucination}). We
therefore propose to verify the format of the enhanced file before running it.

As we saw in the section for \Nref{sec:adsScenarioFormats}, there exists several formats for
\acrshort{ads} scenarios. In order to verify that the syntax of our enhanced test
case is valid, we simply need to apply the syntax rules of our format.

The CommonRoad format is XML-based~\cite[720]{commonRoadOG} and as such we can
to some extent assess the degree of hallucination by parsing the XML structure.
Furthermore, it has an exhaustive Python library with several utilities\footnote{\url{https://pypi.org/user/commonroad/}}.

OpenSCENARIO exists both as XML and a domain-specific language (DSL). If we
utilise the XML version, we can apply the same methodology as for the CommonRoad
format. If using the DSL version, one way
the OpenSCENARIO format can be verified is by using free
online cloud services such as this offering from AVL
\footnote{\url{https://smc.app.avl.com/validation}}. We should however strive for
running a local verification service to \begin{inparaenum}
    \item save time and compute,
    and
    \item preserve data privacy.
\end{inparaenum}
Besides, it is generally a good idea to limit the number of external dependencies\footnote{Note for
    example how LGSVL\cite{lgsvl} was shut down, preventing projects such as DeepScenario of
    \citeauthor{DeepScenario} to be further developed on the original platform.}.

\subsection{Test case running and evaluation}

\subsubsection{Test case runner}\label{sec:testCaseRunner}

The system will automatically run all
our base test cases using an \acrshort{ads} simulator, and collect data points to get a baseline. It
will later also run the mutated \acrshort{llm}-enhanced versions of the base cases.

We have already ran the test cases in their base form. We will now run their
improved versions in order to compare them to see what effect the \acrshort{llm}
enhancement (see \Cref{sec:llmEnhancement}) has had.

For the reasons we have seen in~\Cref{sec:simulatorOverview}, we want to run our
test cases on Carla. It is the best offering as it is open source, under active
development and has a feature rich Python \acrshort{api}.

We record a plethora of datapoints when executing scenarios on the simulator\footnote{This is again
    provided by the Carla software suite. A complete overview of data points is provided in the
    documentation, see e.g.~\url{https://carla.readthedocs.io/en/0.9.15/adv_recorder/}}.

The way the Carla simulator works, one simulator run can be analysed post factum. The entire
scenario execution is stored in a Carla-specific binary format. This binary file can then later be
analysed, extracting various metrics from one run. This saves time not having to run the simulator
more than necessary, and allows for reproducing the metric calculations from the original underlying
binary log file.

\subsubsection{Test case improvement evaluation}\label{sec:testCaseEval}

We saw in \Cref{sec:adsMetrics} that there are several metrics for assessing
\acrshort{ads}. We will use these metrics when evaluating our improvements.

\subsubsection{Test case result reporting}

We will compare the results from running
the baseline unmodified test case and comparing it with the results from
running the \acrshort{llm}-enhanced version and returning to the user. Ideally with
some automatic analysis of the results.

Having ran both the base test case and its enhanced counterpart, we have
results. The results will be stored in \acrfull{csv} files, allowing \begin{inparaenum}
    \item further analysis in Python/Jupyter,
    and
    \item easy translation to \LaTeX~tables for the final report.
\end{inparaenum}

This is the final step of the envisioned pipeline. Where we have our result, and
need to analyse them.

This last step has great opportunities for being scoped up to a fully integrated
test suite which allows for both running test cases and analysing the results in
a \acrfull{gui}. But we should focus on the prior steps for now, only creating a
\acrshort{gui} if there is sufficient time towards the end of the project to
focus on such non-\acrshort{llm} related topics.

Initially, the results will consist of numerical comparison of the
\acrshort{csv}s with regard to the relevant metrics outlined in
\Nref{sec:testCaseEval}.

We need to define what requirement we will use for determining the \textit{result} of a test case
run. Without this, we cannot compare it to other test cases.

\section{Component details}

After having surveyed a broad overview of the details of the \hefe~pipeline, let us now narrow our scope
and focus on the individual components.

% Alt under her er kopiert over fra det gamle "implementation details"-chpt.
% (Og så tilpasset litt, dekrementert heading levels osv )

The implementation is what facilitates doing the actual experiments. For the most part, it follows
what is outlined in the \Nref{sec:hefeComponentOverview}, with some minor practical differences.
What follows will analyse the implementation of the components of the \hefe{} pipeline and explain
more closely in detail not only \emph{what} they do, as that is already covered in the solution
proposal, but \emph{how} they do it, with hands-on code examples.

All code is available on the GitHub repo \href{https://github.com/orjahren/LLM4DD}{\hefe}.

\subsection{Carla interface and scenario utilities -- Thor}

The Thor module is responsible for all things related to the Carla \acrshort{ads}
simulator. It provides the client with several scenario-related utilities, and
is capable of executing the desired scenarios.


Certain of its utilities are simple tools for asserting the liveness of Carla,
such as the \texttt{get\_carla\_is\_up} function, shown in
listing~\ref{lst:odinCarlaHealthCheck}. This function will use the Carla
standard Python library and attempt to connect to the server on its default
port\footnote{I.e. \num{2000}, line \#4 in listing
    \ref{lst:odinCarlaHealthCheck}.}. Note that we refer to the host as simply
\texttt{carla} -- this is possible due to the entire project running
containerised with Docker Compose. Instead of referring to the specific IP
address of the Carla server (typically localhost, if not running it externally),
the Docker system will facilitate this name translation for us.

\lstinputlisting[caption={Excerpt from carla\_interface.py, demonstrating the implementation of a Carla health check.}, label={lst:odinCarlaHealthCheck}, language={Python}]{hefe-listings/carla_interface.py}

This is used both to assert the general liveness of the \hefe~pipeline, and to
verify that the simulator is available before performing experiments. It is
better to detect this illegal state \emph{before} running experiments rather
than during their execution.

Furthermore, it shall also be equipped with functionality for \emph{executing} \acrshort{ads}
scenarios on Carla. This is trivial when using Carla's existing Scenario Runner
module's functionality.

% Rest API - running test cases - Thor
% “Fast API”, Python
% POST a test case to the API. It will be ran on the ‘server’
% RabbitMQ for listening for finished test cases? So that the client knows it can fetch the results?
% Will need UUID for test cases so the correct result can be fetched after it has been ran
% Need to store these somewhere. NoSQL database?
% This component should also accumulate results.
% Huge TODO: What metric are these results?
% Should be containerised (Docker/Podman)

\subsection{LLM interface and prompt applications -- Odin}\label{sec:odinImplementation}

% Rest API - performing LLM enhancement - Odin
% “Fast API”, Python
% Take a base test case as body
% Have some prompt repository
% Apply prompts with LLMs
% Must integrate with LLM. Either locally (Ollama) or remote (some API)
% Look into good LLM agnostic transition layer. E.g. Aisuite
% https://github.com/andrewyng/aisuite
% Should use same UUIDs as outlined above, but suffixed with e.g. “pure” and “tainted”
% Containerized. Docker compose?

The Odin module handles all things \acrshort{llm}. It provides a unified
\acrshort{api} for applying various prompts to scenarios and returning the
enhanced output resulting from having applied the prompt. We have implemented
support for the \acrshortpl{llm} that are available on \begin{inparaenum}
    \item Ollama, and
    \item Gemini
\end{inparaenum}. This allows for testing with \acrshortpl{llm} such as
\begin{inparaenum}\setcounter{enumi}{2}
    \item Mistral \num{7.2}B, and
    \item gemini-2.5-flash
\end{inparaenum}.

\subsection*{LLM interface implementations}
\subsubsection{Gemini integration}

The Gemini integration is quite straightforward, relying on Google's own
\texttt{genai} Python module. Listing~\ref{lst:thorGeminiInterface} renders the
\emph{entire} interface, again highlighting how straightforward this really is.
The one piece of complexity to not is that it requires that the user provides
their own Gemini \acrshort{api} key and has this set as an environment variable
with the proper name. Without this being as it should, the script will crash, as
it would not possible for it to complete the desired \acrshort{llm} enhancement
regardless as long as the \acrshort{api} key is not present.

\lstinputlisting[caption={llm\_api\_interfaces/gemini\_interface.py, The implementation of a Gemini interface for executing prompts.}, label={lst:thorGeminiInterface}, language={Python}]{hefe-listings/gemini.py}

\subsubsection{Ollama integration}

The Ollama integration is a bit more cumbersome. This mostly comes down to it not
using any existing library modules for this specific purpose, instead relying on
using the \texttt{json} and \texttt{requests} modules to implement the desired
functionality from scratch, making it so that we need to handle network IO and
marshalling the \acrfull{llm} response into a fitting return buffer.

Listing~\ref{lst:thorOllamaInterface} renders the
\emph{entire} interface. As we can see, it is not too bad although nowhere near
as clean as the Gemini implementation (\ref{lst:thorGeminiInterface}).

Its complexity arises principally from \num{2} major factors --
\begin{inparaenum}
    \item the already mentioned manual networking, and
    \item having to parse the streamed response
\end{inparaenum}
Furthermore, this code expects that the user already \emph{has} an Ollama
installation running on their host machine. The code provides no means of setup
for this -- that is an entirely external endeavour that is left up to the end user.

Similarly to how the Gemini implementation does it, this will crash if Ollama is
not functioning properly as it would not possible for it to complete the desired
\acrshort{llm} enhancement regardless if Ollama is unreachable.

\lstinputlisting[caption={llm\_api\_interfaces/ollama.py, The implementation of an Ollama interface for executing prompts.}, label={lst:thorOllamaInterface}, language={Python}]{hefe-listings/ollama.py}

\subsection*{Prompts and their associated code}

In this project, the prompts are the instruction to the \acrfull{llm} for
applying the enhancement to the scenario. Quite possibly the most critical piece
of code related to the experiments. They need to take the base scenario as an
input and integrate it into the \acrshort{llm} context, such that it knows what
it shall use as its base to apply enhancements that will decrease the
driveability. For this reason, it also provides certain scenario
utilities\footnote{That architecturally might as well have been integrated in the
    Thor module\ldots}.

\subsubsection{Scenario utilities}

These are essentially quite trivial helpers. Listing \ref{lst:odinScenarioUtils}
renders the core functionality -- hopefully this is quite self-explaining.

\lstinputlisting[caption={scenario\_utils.py, The implementation of various scenario helper functions for executing prompts.}, label={lst:odinScenarioUtils}, language={Python}]{hefe-listings/scenario_utils.py}

\subsubsection{Prompts -- templating and usage}

% TODO: Finnes det noe bigbrain måte å slippe å manuelt måtte referere ti llinjenummer?
As mentioned, the prompts need to include the scenarios \emph{in} them, so that
they are accessible to the \acrshort{llm}. How this is done, is rendered in
listing~\ref{lst:odinPromptTestbed}. The most interesting aspect is how the
prompts are stored in the system as lambda functions. This makes it so that they
can take an argument that represents the scenario --
\texttt{python\_carla\_scenario\_raw}(line \#16) -- and simply \emph{execute} the
function to insert the scenario into the prompt (line \#42). This is then
inserted into the output prompt at the location located at line \#21 in the listing.

% TODO: Linjenummere må oppdateres
% TODO: Må legge til at man også kan passe inn scenario-navn og metrics

\lstinputlisting[caption={experiments/testbed/prompts.py, The implementation of a prompt testbed for executing prompts.}, label={lst:odinPromptTestbed}, language={Python}]{hefe-listings/testbed_prompts.py}

Lastly, note the comments in the top of the file, intended to give GitHub
Copilot increased understanding of the context, so that it can provide better
aid during programming.


\subsection{Execution tool / user oriented frontend -- Loki}

% Client - orchestrating the process - Loki
% Fetch available test cases from Thor? Select what/which are to be used
% Store results clientside? Separate database for this?

The final module of the \hefe~pipeline is Loki -- it is simply a tool intended
to be used by the user for operating the process. It \begin{inparaenum}
    \item  says what scenarios are available to it (i.e. those that are eligible for
    being enhanced), and
    \item allows the user to select a prompt and
    \item execute that prompt to the scenario of their choosing.
\end{inparaenum}

\lstinputlisting[caption={loki/main.py, The implementation of the Loki script.}, label={lst:lokiMainImplementation}, language={Python}]{hefe-listings/loki_main.py}

Listing~\ref{lst:lokiMainImplementation} renders the implementation of the
script. It relies on the Odin and Thor modules for all essential functionality,
which is in line with what is to be expected as this is simply a frontend client
to \emph{reach} them.

It is relies on the \texttt{requests} module for doing \acrfull{rpc} to the
other modules. There is also the outlines of a RabbitMQ implementation, which is
why \texttt{pika} is being imported. As of now, this is in non-functioning
alpha. Implementation of RabbitMQ message passing has not been prioritized as
there were, as mentioned, more important issues to focus on that would yield
better and more important results when resolved. This would maintain feature
parity with the \texttt{requests}-based approach.
\chapter{Experiment methodology}\label{chp:experiments}

\epigraph{The torment of precautions often exceeds the dangers to be avoided. It is sometimes better to abandon one's self to destiny.}{Napoléon}

This chapter will describe the experimentation that has been done with the implemented solution
proposal (\Cref{chp:solutionProposal}), in anticipation of analysing the \Nref{chp:results}.

\section{Prompts}

Prompting is our principal way of interfacing with the \acrshort{llm}. For this reason, our results
rely on \begin{inparaenum}
    \item good, and
    \item fitting prompts
\end{inparaenum}. Without this all is lost.

We therefore propose several prompting strategies, taking after related research
(\Nref{chp:relatedWork}).

Prompts were determined by trial and error in an iterative manner, in
conjunction with GitHub Copilot. They are all descendant of listing
\ref{lst:firstPrompt}, each subsequent iteration improving on the last based on
what worked or did not worked when assessing the output.
Due to a technical detail of the
\hefe~implementation (\Nref{sec:odinImplementation}), the datatype of the prompt is
a lambda function that takes the raw scenario represented as a string and then
inserts it into the prompt in runtime. This is represented by the curly braces
on line \num{3} in listing~\ref{lst:firstPrompt}.

\begin{lstlisting}[caption={The first prompt.}, label={lst:firstPrompt}, language={Python}]
lambda python_carla_scenario_raw: f"""
1 - Context: We are working with a driving simulation environment for the Carla simulator.
2 - Task: Decrease the driveability of the scenario by enhancing it with more details and complexity.
3 - Input: {python_carla_scenario_raw}
4 - Output: An enhanced version of the scenario description with additional
details and complexity, still in Python carla scenario format.
""",
\end{lstlisting}

\section{Finding a suitable LLM}

As we learnt in \Cref{sec:llmJungle}, there are several \acrshortpl{llm} extant. We should
experiment with various different \acrshortpl{llm} to maximize our chance of testing with a `good'
\acrshort{llm} that goes well with our stated purpose.

The experiments were first carried out using a locally hosted \num{7.2}B parameter Mistral model.
This model is interesting in that it has been shown to outperform significantly larger models
across various benchmarks\footnote{\url{https://ollama.com/library/mistral}}. Due to its small
nature, however, these initial results were not that promising. Later, in order to obtain better
results, the Gemini model \texttt{Gemini 2.5 flash} running on Google's infrastructure was used.
This is a mid-size multimodal model that supports up to 1 million tokens, released in June of 2025,
with support for thinking and long
contexts\footnote{\url{https://deepmind.google/models/gemini/flash/}}.

All data in the \Nref{chp:results} chapter, are obtained using the Gemini model.

\section{Metrics}

The way the Carla simulator works, one simulator run can be analysed post factum. The entire
scenario execution is stored in a Carla-specific binary format. This binary file can then later be
analysed, extracting various metrics from one run. This saves time not having to run the simulator
more than necessary, and allows for reproducing the metric calculations from the original underlying
binary log file.

Due to the immense file size of these logs\footnote{Keep in mind that they track all actors in the
    scene over time.}, publishing all our raw files is not feasible.


\part{Conclusion}

\chapter{Results}\label{sec:results}

\epigraph{I have not failed. I've just found 10,000 ways that won't work.}{\textit{Thomas A. Edison}}

Our results show that the initially proposed solution of feeding bare
\acrshort{ads} scenarios represented by Python code into \acrshortpl{llm}, does
not yield any meaningful results. This is caused by various reaons. The
following discusses \begin{inparaenum}
    \item why this is, and 
    \item ways by which it can be remedied in future work \end{inparaenum}.

\section{Output of the \acrshort{llm}}

Depending on the prompt, our results show that it \emph{is} possible to get
reasonable-looking Python out of the \acrshort{llm}. One somewhat annoying
detail is their bent to mark the code as specific syntax, applying a
Markdown-formatted code block indicating both that the output \emph{is} code,
and what language it is in.,to the first and last line of the output (Listing \ref{lst:llmOutputMarkdown}). 
% \begin{lstlisting}[language=Markdown]
\begin{lstlisting}[caption={\acrshort{llm}-generated Python code with Markdown syntax. The bracketed part on line 3 has been added for demonstration purposes.}, label={lst:llmOutputMarkdown}]
```python

[ scenario code ]

```
\end{lstlisting}



Upon manually removing these syntactic artefacts, we can go ahead with executing
the scenario. But as previously mentioned, we are unable get any meaningful
results. This comes down to \begin{inparaenum}
    \item halluciantion of Python code, and 
    \item Carla problems.
\end{inparaenum}

\subsection{Hallucinations in the enhanced scenarios}

The \acrshort{llm} typically seems to be on the right track, outlining something
that \emph{sounds} like a good approach to satisfying our prompt of decreasing
the driveability of the scenario. But in practice, it will often hallucinate
methods that don't exist, or use terms and phrasing that are not valid keywords
in the Carla specificication. This is in line with what was found by e.g.
\citeauthor{autoSceneGen}~\cite[14542]{autoSceneGen} (See \Nref{sec:autoSceneGen} in
\Nref*{sec:relatedWork}).

% TODO: Legge til ekesempler her
% TODO: Er vi sikre på at dette handler om spesifikt hallusinering og ikke noe annet?
\subsubsection{Non-existing methods}

As mentioned, the \acrshort{llm} seems to have the right idea of what it can do
to achieve the stated goal. But the way that it goes about obtaining it, does
not always work. The enhanced scenario code will often call methods that don't
exist. This leads to a runtime exception in the scenario runner when executing
the enhanced scenario. 

% TODO: Legge til ekesempler her
% TODO: Er vi sikre på at dette handler om spesifikt hallusinering og ikke noe annet?
\subsubsection{Non-existing arguments}

In a similar vein to the non-existing methods, non-exisiting \emph{arguments}
were also shown to appear. The \acrshort{llm} could simply call methods that
were already being used, with additional arguments that made semeantic sense,
but that were not a part of the function definition. This also causes runtime
exceptions in the scenario runner.

% TODO: Burde refereree / kildeføre / vise til noe forankring for disse keyword-eksemplene.
\subsubsection{Illeal property keywords}

Another trend we observed was the usage of various keywords that simply don't
exist in the Carla repetoire. Where Carla would recognize the word `snowstorm',
the \acrfull{llm} proposed using the word `blizzard'.


\subsection{Carla crashes with certain scenarios}

There appears to be a bug in Carla version 0.9.15\footnote{Which is the version
employed for this project.} which causes the program to \emph{hard crash} when
executing certain scenarios with metric recording enabled. This has been
reported to the project Github\footnote{By several members of the scientific community, see e.g.
\begin{itemize}\item  \url{https://github.com/carla-simulator/carla/issues/9170} and \item \url{https://github.com/carla-simulator/carla/issues/9152}\end{itemize}}, but as of 2025-09-30 it has not been resolved.
Testing shows that the same scenarios may be ran without crashing when
\textbf{not recording}, but this naturally has severe implications for our
opportunities of obtaining data from the simulation run. The `record' function
of the scenario runner is the crux of measuring the driveability of the
scenario. 


\section{Metrics used for evaluation}

We measure several metrics for evaluating the driveability of the scenario. The
principal is \emph{jerk}.

Due to the above resons with getting the enhanced scenarios to run, there is
however minimal data to bases any qualitative analysis on.

\chapter{Discussion}
\section{Environmental concerns}
Cost/benefit with using \acrshortpl{llm}. Refer back to \Nref{sec:llmProblems}.

While we demonstrated promising results in \Cref{sec:results}, it is important to keep in mind the
environmental cost of using the \acrshortpl{llm} for this purpose. How good should the results need
to be in order to justify using \acrshortpl{llm}?

Perhaps future work can look into obtaining similar results using greener strategies.

\section{Realism in the enhanced scenario}

It is very easy to get bad driveability if your scene is bonkers. But there is no real world
value/practical applicability in these scenarios?

\url{https://www.simula.no/research/reality-bites-assessing-realism-driving-scenarios-large-language-models}

\section{LLM context size}

Hvis man har lange scenarios kan de overgå LLMens kontekst size og så mister man ting?

\section{Python / OpenScenario / DSI}

Con med Python: LLMen kan bruke utdatert syntax / bruke ting som ikke stemmer overens med den
versjonen du vil bruke. De andre er mer "konstante" og mindre sårbare for dete
\chapter{Further work}

\section{\acrshort{llm} aspects}

\subsection{Different promtping strategies}

Overdrivelser? Typ "Det er veldig viktig for meg at du gjør dette fordi da blir
jeg glad"? Vise til litteratur som underbygger sånt.

\subsection{Temperature}
Hallucination.

\subsection{Pretraining?}

\subsection{Retrieval-augmented generation (RAG)}
Context, affordances.

\subsection{More models}

More models more good?

\subsection{Tool calling}

Can give the LLM access to tools, e.g. methods for adding objects etc.

\section{GUI visualisations}

Maybe: Frontend client - web GUI - Ivar
If Loki does its job effectively, we can create a web based frontend for doing the process. It could do the same as Loki, but with greater ease of use.
Having a GUI allows for making neat visualisations.
Motivate why our enhanced test cases are better by showing it.

\section{Instant validation of test case syntax}
Compiler-stuff. Syntax. Parsing.

% Og: Verifisere at spawn locations ikke overlapper/kolliderer

\section{Other datasets}

We used dataset x for our experiments. Scenario datasets y and z can also be used

\chapter{Conclusion}

\epigraph{This inductively justifies the conclusion that induction cannot justify any conclusions}{David Deutsch}

In this master's thesis, we propose a tool -- \hefe -- for using \acrfullpl{llm} to decrease the
driveability of \acrfull{ads} scenarios in order to expose underlying weaknesses in the
\acrshort{ads}. We show this work is in line with other works in the field, and we show that our
results are TODO.


\backmatter{}
\printbibliography{}

\appendix
\part*{Appendix}
\chapter{Scenario file diffs}\label{sec:fileDiffs}

The diffs represent the \emph{diff}erence between two files, highlighting what
has changed. In this context -- the red indicates something that was changed
from the original scenario, and the green indicates something that was added by
the \acrshort{llm}. The lines in black are unchanged.

\section{Cut\_in-enhanced-5.py }

\lstinputlisting[caption={The diff of an \acrshort{llm}-enhanced Cut\_in scenario, highlighting \emph{how} the \acrshort{llm} enhanced the scenario.}, label={lst:llmOutputDiff}, language={diff}]{experiment-material/experiment-5.diff}


% Lar være å vise glossaries siden sidetallene er feil, se #5.
% \printglossaries{}
\end{document}
