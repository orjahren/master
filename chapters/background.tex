\chapter{Background}\label{chp:background}

\epigraph{The limits of my language mean the limits of my world.}{Wittgenstein}

This chapter will give an introduction to \acrshortpl{ads} and \acrshortpl{llm},
laying the foundation for understanding the motivation of the project more
in-depth, such that the appeal of the later solution proposal is more clear. 


% \subsection{Autonomous systems}
% 
% Autonomous systems are systems that are capable of changing their behaviour in
% response to unanticipated events during
% operation~\cite[368]{watson2005autonomous}. Autonomous systems exist in several
% different forms and they are typically tailored for their specific operating
% environments. Some systems operate in the maritime domain, some in the air, and
% others on the ground in the terrestrial operating domain~\cite[369-370]{watson2005autonomous}.

\section{Autonomous driving systems (ADSs)}

\acrfull{ads} are systems that enable automotive vehicles to drive autonomously. Due to the typical
operating scenarios of a car it is pivotal that the \acrlong{ads} maintain a high safety standard. A
common way to assert safety is to use simulator based testing~\cite[1]{DeepScenario}.

\subsection{Autonomous driving system driveability}\label{sec:adsDrivability}

\emph{Driveability} is a high-level estimator of the overall driving
condition of an \acrshort{ads}, derived from several lower-level sources~\cite[3140]{safeToDrive}.
It can be used to refer to various aspects of a scene.
\citeauthor{safeToDrive} discuss the concept further, using the  scene definition
of~\citeauthor{scenes} as outlined in  \Cref{sec:adsSimConcepts}, they describe
how driveability can refer both to \begin{inparaenum}
    \item road conditions, and
    \item human driver performance.
\end{inparaenum}
\citeauthor{safeToDrive} go on to give an overview of how driveability
can be used to refer to a \begin{inparaenum}\setcounter{enumi}{2}
    \item \textit{driveability map} which divides a map into
    cells indicating where the \acrshort{ads} expects that it will be able to go, and
    \item \textit{object driveability}, which refers to the classification of physical objects in
    the environment that the \acrshort{ads} expects that it can run over without causing damage to
    the ego vehicle~\cite[3135-3136]{safeToDrive}.
\end{inparaenum}

The main method for assessing the driveability of a scene comes form assessing the environment of
the scene. Factors such as \begin{inparaenum}
    \item weather,
    \item traffic flow,
    \item road condition, and
    \item obstacles \end{inparaenum} all play into this. The \acrshort{ads} infers information from
observation~\cite[3136]{safeToDrive}.

They continue to give an overview of various \textit{driveability factors} and their associated
difficulties, using a a split between \textit{explicit} and \textit{implicit} factors.

\emph{Explicit driveability factors} will typically include factors such as \emph{Extreme
    weather} such as \begin{inparaenum}
    \item fog,
    \item heavy rain,
    \item snow,
\end{inparaenum}
all serving to impair road visibility and causing increased difficulties for vision-based tasks such
as road detection and object tracking~\cite[3136-3137]{safeToDrive}. \emph{Illumination} also
poses various challenges for typical \acrshort{ads} tasks as a typical \acrshort{ads} will be
required to operate in a plethora of scenes with varying degrees of illumination depending on
factors such as time of day and location (e.g. if the \acrshort{ads} is operation in a dimly lit
tunnel)~\cite[3137]{safeToDrive}. The authors highlight how low illumination may serve as an
advantage for the \acrshort{ads} as this allows for using the head lights of other vehicles as a
feature for detecting them, whereas it make pedestrian detection significantly more
challenging~\cite[3137]{safeToDrive}. \emph{Road geometry} is another external factor, satisfying
our natural intuition that \textit{intersections} and \textit{roundabouts} are more difficult to
drive through than straight highways~\cite[3137]{safeToDrive}.

\emph{Implicit driveability factors} consist of behaviours and intent of other road users
interacting with the autonomous car~\cite[3138]{safeToDrive}. This includes the actions of other
vehicles such as their \begin{inparaenum}
    \item overtaking,
    \item lane changing,
    \item rear-ending,
    \item speeding, and
    \item failure to obey traffic laws \end{inparaenum}.~\citeauthor{safeToDrive} call these factors
\emph{vehicle behaviours}~\cite[3138]{safeToDrive}. Furthermore, \emph{pedestrian behaviours}
are also taken into account, noting how pedestrians can sometimes
\begin{inparaenum}\setcounter{enumi}{5}
    \item cross the road,
    \item be inattentive, or
    \item fail to comply with the traffic law \end{inparaenum}\cite[3138]{safeToDrive}. They go on
to describe the \emph{driver behaviour} of other drivers pointing out how
\begin{inparaenum}\setcounter{enumi}{8}
    \item distraction, and
    \item drowsiness \end{inparaenum} can be factors that cause accidents even for
\acrshort{ads}-enhanced vehicles due to the other, manual, cars
interfering with their operation~\cite[3138-3139]{safeToDrive}. Lastly
\emph{motorcyclist/bicyclist behaviours} cause their own source of implicit driveability factors:
The models and methods developed for analysing the group's behaviour are far more limited than other
groups of road users~\cite[3139]{safeToDrive}.~\citeauthor{safeToDrive} theorise that this comes
down to the lack of available datasets that capture and label the trajectories and behaviours
of motorcyclists and bicyclists~\cite[3139]{safeToDrive}, causing potential issues for any
\acrshort{ads} that wishes to operate in a shared traffic environment with this group.

\section{Autonomous driving system testing}

Testing is essential for assuring \acrlong{ads} operative safety~\cite[163]{ADTestingReview16}.
Several methods for testing exist, testing various aspects of the \acrlong{ads}. An \acrshort{ads}
typically exists of several modules, all working together and handling different aspect of the
\acrlong{ads}.

\citeauthor{ADTestingReview16} outline several typical architectures for \acrshort{ads} testing,
drawing on traditional software testing traditions outlining how \textit{software testing} can be
used alongside more specialized \acrshort{ads} testing techniques such as \textit{simulation
    testing} and \textit{X-in-the-loop testing}~\cite[163-164]{ADTestingReview16}.


\subsection{Autonomous driving system testing metrics}\label{sec:adsMetrics}

When evaluating \acrshort{ads} testing, several metrics can be used. What metric to use will depend
on what the relevant test is measuring.

Building on what we have learnt about driveability (\Cref{sec:adsDrivability}),
we take after \citeauthor{safeToDrive} and review three metrics for
quantifying driveability: \begin{inparaenum}
    \item scene driveability,
    \item collision-based risk, and
    \item behaviour-based risk. Finally, we also investigate the mertric
    \item jerk.
\end{inparaenum}

\emph{Scene driveability} refers to how easy a scene is for an
\acrshort{ads} to navigate, and the \textit{scene driveability score} refers to
how likely the \acrlong{ads} is to fail at traversing the
scene~\cite[3140]{safeToDrive}. It is typically found through and end-to-end
approach. Note how this is a metric for \textit{scenes}, without taking into
account the performance of any specific \acrshort{ads}.

\emph{Collision-based risk} comes in two kinds - \begin{inparaenum}
    \item binary risk indicator, and
    \item probabilistic risk indicator.
\end{inparaenum} \citeauthor{safeToDrive} posit that the prior, binary metric, indicates whether a
collision will happen in the near future in a binary `either-or' sense, whereas the latter yields a
probability calculated based on current states, event, choice of hypothesis, future states and
damage~\cite[3140]{safeToDrive}.

\emph{Behaviour-based risk} estimation also represents a binary classification problem wherein
nominal behaviours are learnt from data, and then dangerous behaviours are detected on that. This
requires a definition of `nominal behaviour', which is typically defined on on acceptable speeds,
traffic roles, location semantics, weather conditions and/or the level of fatigue of the
driver~\cite[3140]{safeToDrive}. Furthermore \citeauthor{safeToDrive} describe how this metric also
allows more than one \acrshort{ads} to be labelled as `conflicting' or `not
conflicting'~\cite[3140]{safeToDrive}, representing a ruling on their compatibility. Finally, they
note how behaviour-based risk assessment typically focuses on driver behaviours, not taking into
account other actors in the scene such as pedestrians or cyclists.

Furthermore, \emph{jerk} is a metric that renders the change of vehicle acceleration with respect to
time. It has been used been used as a measure of the smoothness or abruptness of a movement in many
domains such as the trajectory planning of the human arm and industrial
robots~\cite[126]{fengJerk17}. Jerk has also been shown to relate to a driver’s
physiological feelings of ride comfort~\cite[126]{fengJerk17}, giving it a clear relation to our
previously stated definition of driveability.~\citeauthor{fengJerk17} go on to posit that a goal of
driving should be to minimize the jerk, as it both is both \begin{inparaenum}
    \item linked to comfort, and
    \item detection of safety-critical events
\end{inparaenum}\cite[126]{fengJerk17}.

\subsection{The complexities of ADS testing}\label{sec:adsTestingComplexity}

As we have seen, \acrshortpl{ads} can perform several tasks, in several environments. As such, there
are several relevant factors for testing them. It is not feasible to test all potential variations
of all potential environments in the real world, meaning that the \textit{test
    coverage}\footnote{See \Nref{sec:testCoverage}} typically will be low.

Some of the factors that complicate \acrshort{ads} operations are \begin{inparaenum}
    \item timing,
    \item sequence of events, and
    \item parameter settings such as the different speeds of various vehicles and other actors.
\end{inparaenum}

\citeauthor{adsComplexityIndex18} posit that \textit{the concept of complexity exists everywhere,
    but there is no agreement on one for driving situations}~\cite[1182]{adsComplexityIndex18}.
Therefore they introduce their own concept of \acrfull{dsc}, which serves to give a metric of a
the complexity of a given driving situation. Their \acrshort{dsc} is defined as the output of a
mathematical formula taking into account the perplexity and standard deviation of several
control variables $\mathcal{M}$ representing the surrounding vehicle's
behaviour~\cite[1182]{adsComplexityIndex18}. Their formula also takes into account the ratio of
\textit{V2X}-capable vehicles~\cite[1182]{adsComplexityIndex18}, i.e. the vehicles that are
connected and capable of communicating~\cite[1]{v2xTestingSurvey2019}.

% It is common to modularize the testing of \acrshort{ads} so that the individual modules can be
% tested in isolation. The modules of an \acrshort{ads} typically include a motion planner, a
% \acrfull{cv} system, and % LIDAR. This allows for testing the individual modules in a way that
% makes sense for their specific domains.

\subsection{ADS simulation}

Due to the complexity involved in testing \acrlongpl{ads} (\Cref{sec:adsTestingComplexity}),
simulators are typically used for this purpose~\cite{DeepScenario}. While the same points about not
being able to test \textit{all} possible scenarios do remain true for simulator based testing due to
the sheer number of factors, using a simulator allows for far greater testing at far lower cost due
to the minimal overhead of
\begin{inparaenum}
    \item generating,
    \item running, and
    \item evaluating the outcome of
\end{inparaenum}
test cases.

Furthermore, simulators allow for greater flexibility in determining the test scenarios due to not
being confined by the  physical world that is available to the scientist that wishes to perform the
testing. Using a simulator, a Europe-based scientist can test their \acrshort{ads} for North
American conditions, or vice-versa.

\subsection{The ADS simulator jungle}\label{sec:simulatorOverview}

Due to the appeal of running \acrshort{ads} simulation, several contenders exist
on the market.

\emph{Carla} is a widely used \acrshort{ads} simulator~\cite{Carla}. It is implemented
using the game engine UnrealEngine~\cite{unrealengine} and allows for running
test cases under various scenarios and collecting their results. Carla is fully
open source and is under active development. It has been applied in projects such as KITTI-Carla,
which generated a KITTI dataset using Carla~\cite{kittiCarla}.

\emph{LGSVL} is a deprecated simulator from LG~\cite{lgsvl}. It was used in projects such
as DeepScenario~\cite{DeepScenario}. It allowed for running various maps with various vehicles and
tracking their data. It was also capable of generating HD
maps \footnote{\url{https://github.com/lgsvl/simulator?tab=readme-ov-file\#introduction}}.
DeepScenario is a project similar to this, concerned with testing \acrlongpl{ads}. Further details
about it in are located in \Nref{sec:relatedWork}.

\emph{AirSim} is Microsoft's offering~\cite{airsim}. It has, like LGSVL,
been deprecated. It is also built using UnrealEngine. Unlike the other
simulators we have seen, this also focused on autonomous vehicles outside of
only cars, such as drones.


\subsection{Concepts of ADS simulation}\label{sec:adsSimConcepts}

\citeauthor{scenes} draw up an outline for the terms \textit{scene}, \textit{situation}, and
\textit{scenario}, that are all concepts widely used in \acrshort{ads} simulation testing.

\emph{scene} is a term that is used in different manners in various
articles~\cite[982]{scenes}, but \citeauthor{scenes} propose standardising the definition on
\textit{a scene describing a snapshot of the environment including the scenery and dynamic elements,
    as well as  as all actors’ and observers’ self-representations, and the relationships among those entities}~\cite[983]{scenes}.

\emph{situation} is, like \textit{scene}, employed in various fashions. \citeauthor{scenes}
give a background detailing its usage ranging from \textit{"the entirety of circumstances,
    which are to be considered by a robot for its selection of an appropriate behaviour pattern in a
    particular moment'}\footnote{The translation from German is borrowed from \citeauthor{scenes},
    \cite[984]{scenes}}, in  \citeauthor{scenarioTysk}~\cite[3]{scenarioTysk} to
\citeauthor{schmidtScenario} introducing a distinction between \textit{the true world} in a formal
sense, and that being the ground truth upon which a situation is
described~\cite[892]{schmidtScenario}.

\citeauthor{scenes} propose to standardise on the definition of a situation being \textit{
    the entirety of circumstances, which  are to be considered for the selection of an
    appropriate behaviour pattern at a particular point of time}~\cite[985]{scenes}.

\emph{scenario} refers to \textit{'the temporal development between several scenes in a sequence
    of scenes'}\cite[986]{scenes}. We note how the definition a a scenario utilises that of a scene.
Furthermore, \citeauthor{scenes} hold it to be the case that \textit{'every scenario starts with an
    initial scene. Actions \& events as well as goals \& values may be  specified to characterize
    this temporal development in a scenario'}~\cite[986]{scenes}, clarifying the distinction
between a scenario and a scene.

Lastly they posit that a scenario spans a certain amount of time, whereas a scene has no such
temporal aspect to it.


When running a simulation, we refer to the autonomous vehicle that is being
simulated as the \textit{ego vehicle}~\cite{egoDefinition}.

% introduce the concept of a \textit{scene}, which denotes 

% An alternate simulator is LG SVL~\cite{lgsvl}, but it has been deprecated since
% 2022 and as such it does not seem proper to build a new solution on top of it.
% The DeepScenario dataset~\cite{DeepScenario} utilises this simulator framework.
% \tanke{The LGSVL part can be removed - not relevant?}

\subsubsection{ADS scenario formats}\label{sec:adsScenarioFormats}

\emph{OpenSCENARIO} is a standard developed by the Association for Automation and
Measurement Systems (ASAM), which is dedicated to the description of dynamic
scenarios~\cite[651]{generatingOpenScenario}. Under this format, only the
\textit{dynamic} content of the scenario is recorded in the file. The static
content is kept in other formats such as OpenDRIVER and
OpenCRG~\cite[652]{generatingOpenScenario}. The simulator Carla (outlined in
\Cref{sec:simulatorOverview}) supports this
standard~\cite[652]{generatingOpenScenario}.

Another widely popular scenario format is
\emph{CommonRoad}~\cite[4941]{convOpenScenarioToCR}, first proposed in
\citedate{commonRoadOG}~\cite{commonRoadOG}. There are tools such as those
proposed by \citeauthor{convOpenScenarioToCR} that allows for converting
OpenSCENARIO scenarios to the CommonRoad
format~\cite[4941]{convOpenScenarioToCR}.

% TODO: Skrive om Carlas støtte for Python-scenarioer
% TODO: Skrive om Carla scenario runner her?

\section{Large language models (LLMs)}

\acrfullpl{llm} are transformer-based language models that typically contain several hundred billion
parameters and are trained on massive text data~\cite[4]{llmSurvey}.
Base language models, as the name implies, \textit{model language}. They are typically statistical
models and an example of \acrfull{ml}.

\subsection{Architecture}\label{sec:llmArch}

A \acrlong{llm} is a neural network trained on big data~\cite[3]{llmSurvey}. They expand on the
older statistical language models by training on more data. This gives rise to \textit{emerging
    abilities} such as in context learning~\cite[3]{llmSurvey} (\Nref{sec:emergentAbilities}). These
older statistical models are also neural networks, but they were impractical to train on large
amounts of data. It was not until the seminal paper \textsc{Attention is all you
    need}~\cite{attentionIsAllYouNeed} that a Google team headed by~\citeauthor{attentionIsAllYouNeed}
showed how neural networks can be trained in parallel using their new \textit{attention} mechanism.
This allowed for using amounts of data that was not technologically practical up until that point,
opening the door for later advancements such as
ChatGPT~\cite[9]{llmSurvey}

% \todo{Should elaborate further?}
\citeauthor{jm} describe how \acrshortpl{llm} rely on \textit{pretraining}.

\subsubsection{The importance of training data}

As a consequence of \acrshortpl{llm} being statistical models of a certain input
data~\cite[1]{llmSurvey}, what data the model is trained on is of great
importance for the capabilities of the model~\cite[6]{llmSurvey}.
\citeauthor{llmSurvey} give an overview of various \acrshortpl{llm} and what
kinds of corpora\footnote{A corpus (pl. corpora) refers to a document
    collection.} they have been trained on~\cite[11-14]{llmSurvey}.

The training data will provide the model with its base understanding of the
world, and as such it will dictate \begin{inparaenum}
    \item what it `knows', and
    \item how we should interact with it.
\end{inparaenum}
E.g., if we want to solve problems related to software code, we should employ a
model that has been \textit{trained} on software code related topics so that the
probability of it predicting correct tokens will be higher. If it has not seen
any code during its training it would not have any base `knowledge' for solving
our problem, and its output would be bad. The \acrshort{llm} would however have
no way of knowing if its output would be right or wrong, and we could say that
it would have \textit{hallucinated}.
See \Nref{sec:llmProblems} for further information
about hallucination.


\subsection{Emergent abilities}\label{sec:emergentAbilities}

\citeauthor{emergentabilitiesLLM} outline how \textit{emergent abilities} appear
when scaling up language models~\cite[1]{emergentabilitiesLLM}. They define
\textit{emergent ability} to refer to abilities that are not present in smaller
models, but present in the larger ones\cite[1]{emergentabilitiesLLM}, building
on physicist~\citeauthor{anderson1972more} stating that \textit{Emergence is
    when quantitative changes in a system result in qualitative changes in
    behaviour.}~\cite[2]{emergentabilitiesLLM}.

Furthermore, they discuss how \textit{few-shot prompting} typically can achieve
far superior results for harvesting \acrshort{llm} emergent abilities, whereas
one-shot prompting can perform worse than randomized
results~\cite[3-4]{emergentabilitiesLLM}.

They continue outlining several approaches for achieving augmented prompting
strategies, underlining how \begin{inparaenum}
    \item multi-step reasoning
    \item instruction following
    \item program execution,
    and
    \item model calibration
\end{inparaenum}
all serve as possible ways of increasing \acrshort{llm} performance~\cite[5]{emergentabilitiesLLM}.

% \tanke{Burde bygge på forrige seksjon om "små" language models}

\subsection{Intelligence in LLMs}\label{sec:llmIntelligence}

There are three theories on machine intelligence, each serving to
explain how they `\textit{think}': \begin{inparaenum}
    \item stochastic parrot
    \item Sapir-Whorf hypothesis,
    and
    \item conceptual blending.
\end{inparaenum}

\subsubsection{Stochastic parrot}\label{sec:llmParrot}

\citeauthor{parrot} outline how \acrshortpl{llm} can \textit{fool} humans as they
are trained on ever larger amounts of parameters and data, appearing to be in possession of an
intelligence~\cite[610-611]{parrot}.

This anticipates the phenomenon of hallucination (\Cref{sec:llmHallucination}).

\subsubsection{Sapir-Whorf hypothesis}

The Sapir-Whorf hypothesis posits that  \textit{The structure of anyone’s native
    language strongly influences or fully determines the world-view he will acquire
    as he learns the language.}~\cite[128]{sapirWhorf}.

We note how this maps to our \acrshortpl{llm}, indicating that they will only ever
be able to `know' the data on which they have come into contact with.

Or: \emph{Language} defines the possible room for \emph{thought}.


\subsubsection{Conceptual blending}
%Relevant?

Conceptual blending is a theory on intelligence. It refers to the basic mental
operation that leads to new meaning or insight that occurs when one identifies
a match between to input mental spaces, to project selectively from those inputs
into a new `blended' mental space~\cite[57-58]{conceptBlending}.

This phenomenon explains how we are able to imagine phenomena that logically
should not exist such as \textit{land yacht} (\Nref{fig:landYacht})

\begin{figure}[h]
    \centering
    \includegraphics[width=0.8\textwidth]{figures/landYacht.png}
    \caption[Land yacht conceptual blend]{The conceptual blend of a \textit{land
            yacht}\footnotemark}\label{fig:landYacht}
\end{figure}

\footnotetext{Diagram borrowed from \citeauthor{conceptBlending},~\cite[67]{conceptBlending}.}

We note how this is how \acrshortpl{llm} operate when processing vectorized
linguistic data.
% Conceptual blending is a theory on both machine and human intelligence. 


\subsection{Utilising LLMs - Prompt engineering}\label{sec:llmUtilization}


A typical way of interacting with \acrshortpl{llm} is \textit{prompting}~\cite[44]{llmSurvey}. You
prompt the model to solve various tasks. As we saw in \Nref{sec:emergentAbilities}, the level of
performance you are able to extract from your \acrlong{llm} can depend a great deal on how you
interact with it. The process of manually creating a suitable prompt is called \emph{prompt
    engineering}~\cite[44]{llmSurvey}.~\citeauthor{llmSurvey} outline three principal prompting
approaches:

\emph{\acrfull{icl}} is a representative prompting method that formulates the task
description and/or demonstrations in natural language text~\cite[44]{llmSurvey}. It is based on
\textit{tuning-free prompting} and it, as the name implies, never tunes the parameters of the
\acrshort{llm}~\cite[15]{promptingSurvey}. One the one hand, this allows for efficiency, but on the
other hand, heavy engineering is typically required to achieve high accuracy, meaning you must
provide the \acrshort{llm} with several answered prompts~\cite[16]{promptingSurvey}. In layman's
terms, \acrshort{icl} entails including examples of the process you want the model to perform when
prompting it.

\emph{\acrfull{cot} prompting} is proposed to enhance \acrlong{icl} by involving a
series of intermediate reasoning steps in prompts~\cite[44, 52]{llmSurvey}. The basic concept of
\acrshort{cot} prompting, is including an actual \acrlong{cot} inside the prompt that shows the way
form the input to the output~\cite[52]{llmSurvey}.~\citeauthor{llmSurvey} note that the same effect
can be achieved by including simple instructions like `\textit{Let's think step by step}' and other
similar `magic prompts' in the prompt to the \acrshort{llm}, making \acrshort{cot} prompting easy to
use~\cite[52]{llmSurvey}.

\emph{Planning} is proposed for solving complex tasks, which first breaks them down into smaller
sub-tasks and then generates a plan of action to solve the sub-tasks one by
one~\cite[44, 54]{llmSurvey}. The plans are being generated by the \acrshort{llm} itself upon
prompting it, and there is a distinction between text-based and code-based approaches. Text-based
approaches utilise natural language, whereas code-based approaches utilise executable computer code~\cite[54-55]{llmSurvey}.


\subsection{General challenges with LLMs}\label{sec:llmProblems}

We have seen that \acrshortpl{llm} demonstrate promising abilities (\Nref{sec:emergentAbilities}) But they have nevertheless certain issues attached to them that we need to be aware of.

\subsubsection{Hallucination}\label{sec:llmHallucination}

As we saw in \Cref{sec:llmParrot}, \acrshortpl{llm} are prone to
\textit{bullshitting}. They have no intuition of, or concern with \textit{the
    truth}. They only ever yield whatever response is the most probable under their
\textsc{beam search} algorithm being applied on their training data.

\subsubsection{Environmental concerns}

A University of Rhode Island study on the environmental impact of \acrshortpl{llm} have shown that
they require wast amount of energy and water~\cite{hungryLlm}. They also found that the different
\acrshortpl{llm} may differ greatly in their energy consumption, highlighting that that certain
\acrshortpl{llm} may consume more than \num{70} times more energy than others~\cite{hungryLlm}.

Another study by \citeauthor{llmCarbon} focusing specifically on \textit{carbon emissions} did
however find that these emissions significantly lower for \acrshortpl{llm} than humans for specific
tasks such as text and image generation, ranging from \num{130} to \num{2900} times less Co2 emitted
depending on the task~\cite[1]{llmCarbon}.

\citeauthor{thirstyLlm} surveyed the water consumption of \acrshortpl{llm}, finding that training the
\acrshort{llm} \textsc{GPT-3} could evaporate as much as \num{700000} litres of clean
freshwater~\cite[1]{thirstyLlm}. Furthermore they review the trends of current AI adoption and
project that the water consumption of AI could reach levels as high as \num{4.2} - \num{6.6} billion
cubic metres by \num{2027}, which is comparable to \num{4} - \num{6} Denmarks, or half of the United
Kingdom~\cite[1]{thirstyLlm}. Recent research indicates that \textit{serving} \acrshortpl{llm}
currently account for more emissions than training them~\cite[37]{sustainableLlmServing}.

Efforts to achieve greener \acrshortpl{llm} have been proposed by \citeauthor{sproutGreenLlm}, while
recognizing the trade-off between ecological sustainability and high-quality
outputs~\cite[21799]{sproutGreenLlm}.

% TODO: Include or not?
% \subsubsection{Cognitive atrophy}
% https://arxiv.org/abs/2506.08872 

% TODO: Include or not?
% \subsubsection{LLM collapse}
% https://machinelearning.apple.com/research/illusion-of-thinking 


\subsection{The different kinds of LLMs}\label{sec:llmJungle}

There are several available \acrshortpl{llm}, some of which are open source, and some proprietary.
Open source \acrshortpl{llm} afford greater insight into their composition and underlying training
data, whereas proprietary models appear more like black boxes. Some popular model families include
the GPTs, Gemini, Llama, Claude, Mistral, and DeepSeek.

The \acrshortpl{llm} differ primarily in their \begin{inparaenum}
    \item parameters, and
    \item training data.
\end{inparaenum}
As we saw in \Cref{sec:llmArch}, all typical \acrshortpl{llm} utilise a transformer-based neural
network. But due to their various different properties, different models can behave differently for
different tasks regardless of their similar architecture.

What they all share is their ability to perform \textit{inference}, meaning that they predict output
tokens given some input tokens (see \Cref{sec:llmParrot}).

\subsection{Existing LLM applications for ADS}\label{sec:llmsForAds}


\citeauthor{LLM4AD} give a broad overview of some of the ways \acrshortpl{llm} have been applied for
\acrshortpl{ads}, highlighting some of the opportunities and potential weaknesses of \acrshort{llm}
applications for \acrshort{ads} purposes. One of the ways \acrshortpl{llm} can be applied, is for
adjusting the driving mode, or aiding in the decision-making
process~\cite[1]{LLM4AD}.~\citeauthor{driveAsYouSpeak} delve further into these aspects in their
other work \citetitle{driveAsYouSpeak}, providing a framework for integrating \acrlong{llm}'s
\begin{inparaenum}
    \item natural language capabilities,
    \item contextual understanding,
    \item specialized tool usage,
    \item synergizing reasoning, and
    \item acting with various modules of the \acrshort{ads}
\end{inparaenum}~\cite[1]{driveAsYouSpeak}.