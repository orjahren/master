
\chapter{Discussion}

\epigraph{Your scientists were so preoccupied with whether they could, they didn't stop to think if they should}{Dr.~Ian Malcolm}


\section{Environmental concerns}
Cost/benefit with using \acrshortpl{llm}. Refer back to \Nref{sec:llmProblems}.

While we demonstrated promising results in \Cref{sec:results}, it is important to keep in mind the
environmental cost of using the \acrshortpl{llm} for this purpose. How good should the results need
to be in order to justify using \acrshortpl{llm}?

Perhaps future work can look into obtaining similar results using greener strategies.

\section{Realism in the enhanced scenario}

It is very easy to get bad driveability if your scene is bonkers. But there is no real world
value/practical applicability in these scenarios?

\url{https://www.simula.no/research/reality-bites-assessing-realism-driving-scenarios-large-language-models}

Virker som at \cite{LLMScenarioChang24} har gjort et arbeid med å definere metrics for dette.

Kan se dette både opp mot sim2real-gap -- \emph{og} ikke


\section{LLM context size}

Hvis man har lange scenarios kan de overgå LLMens kontekst size og så mister man ting?

Noen scenarios deler samme fil (Accident\_1 m/venner i srunner/scenarios/route\_obstacles.py )

\section{Python / OpenScenario / DSI}

Con med Python: LLMen kan bruke utdatert syntax / bruke ting som ikke stemmer overens med den
versjonen du vil bruke. De andre er mer "konstante" og mindre sårbare for dete