
\chapter{Discussion}\label{sec:discussion}

\epigraph{Your scientists were so preoccupied with whether they could, they didn't stop to think if they should}{Dr.~Ian Malcolm}

The following wil analyze the \Nref{sec:results} in light of what they achieve and evaluate the
fesability of the \hefe~approach in comparison to other methods.

% Fokus her: Hefes value add
% \section{Dette kontra tradisjonell generering}
\section{Scenario modification versus scenario generation}

This project concerns itself with taking existing scenarios and \emph{modifying} them. This is in
many ways similar to scenario \emph{generation}, but there are also certain key differences. Let us
review some of these.

% TODO: Gir det mening å ha dette som en subsection når det bare er én i seksjonen?
\subsection{Unknown substrate}

When generating a new scenario, the principal factor in determining what it will contain, is a
combination of your generation technique, and what sort of training data you will have used for
obtaining this technique. As such, you might not always have complete control of what underlying
data will be used for the generation of your specific scenario.

With the \hefe~approach wherein the user provides a base scenario themself, there is \num{100}\%
certainty of what the base scenario will be. Thus, we obtain significant inherent knowledge of the
scenario substrate. Naturally, you may object, the same problem of the `unknown substrate' will
present itself in form of our \acrshort{llm} executing the prompt and inferring what the new --
enhanced -- scenario will look like. But then, I maintain that we still propose a significant value
in the increased awareness \emph{of} the base scenario and its substrate. As we have seen in the
\Nref{sec:results}, the changes made by the \acrshort{llm} typically don't alter the underlying
ontology of the scene.

This touches on a broader aspect concerning the \emph{value} of having this knowledge of the
scenario substrate. While this is not the focus of this work, one potential aspect could be a sort
of grounding related to the `sim2real' gap, and the realism of the scene. Let us now delve further
into this.

\section{Realism in the enhanced scenario}

If your task is to `obtain bad driveability in a scenario using \acrlongpl{llm}' in a very general
sense, one can imagine all sorts of creative ways this can be achieved. But in our more narrow scope
of wishing to highlight practical faults in the \acrshort{ads}, we must add another criterium --
realism.

If your scene is bonkers, it will be very easy to get bad driveability. But there is little value
and/or practical applicability in these scenarios. Realism in \acrshort{ads} siumulator scenarios is
a research field in itself with major implications for how scenario-based testing ought to be done.

\citeauthor{RealityBites} have looked into evaluating \acrshort{ads} simulator scenario using
\acrshortpl{llm}~\cite[40]{RealityBites}. Note the distinction between this sort of `realism' in
this ` aligns with our understanding of reality'-approach, and the more technical understanding of
realism that posits that the scenario must adhere to the laws of physics, and make sure to not spawn
objects in such a way that they intersect (which would also violate the laws of physics).
\citeauthor{LLMScenarioChang24} also underly the importance of realism, proposing a scoring function
that takes realism into account for evaluating their \acrshort{llm}-generated \acrshort{ads}
simulator scenarios~\cite[6581-6582]{LLMScenarioChang24}.


Kan se dette både opp mot sim2real-gap -- \emph{og} ikke

% Ontologi, substrat, arv 

% Modellen kan være trent på lignende scenarios så selv om man genererer vil det
% til dels ha samme substrat 

% Ontologisk summering -> ett scenario som tilføyes informasjon fra flere andre
% scenarios? 

% Perks: Man har mer kontroll over utgangspunktet, så man er viss på at det man
% får ut vil være forankret til den virkeligheten man får presentert i det
% opprinnelige scenarioet. Kontraster til paperet som bruker LLM for å \emph{søke} i
% eksisterende scenario og så bare bruke det som matcher mest, der vil man ikke nødvenidgivs ha
% tilsvarende kontroll på hvilken base man bruker og at denne tilfredsstiller
% relevante krav.


\section{Environmental concerns}
Cost/benefit with using \acrshortpl{llm}. Refer back to \Nref{sec:llmProblems}.

While we demonstrated promising results in \Cref{sec:results}, it is important to keep in mind the
environmental cost of using the \acrshortpl{llm} for this purpose. How good should the results need
to be in order to justify using \acrshortpl{llm}?

Perhaps future work can look into obtaining similar results using greener strategies.



\section{LLM context size}

Hvis man har lange scenarios kan de overgå LLMens kontekst size og så mister man ting?

Noen scenarios deler samme fil (Accident\_1 m/venner i srunner/scenarios/route\_obstacles.py )

\section{Python / OpenScenario / DSI}

Con med Python: LLMen kan bruke utdatert syntax / bruke ting som ikke stemmer overens med den
versjonen du vil bruke. De andre er mer "konstante" og mindre sårbare for dete

\section{Alignment}

Hvordan kan vi være sikre på at LLMen ikke nekter å generere visse typer scenarioer? Kan det være
bias i hva den tillater? Og ihva den gjør??