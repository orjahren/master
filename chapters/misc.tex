\chapter*{Abstract}

\acrfullpl{ads} rely on extensive testing in order to verify their operational safety. But due to
their nature of being able to operate in any unseen environment with arbitrary external actors,
the number of potential scenarios is infinite. It is therefore important to obtain the least
driveable scenarios for simulator testing ahead of real world deployment. We therefore propose
applying \acrlongpl{llm} to \acrlong{ads} scenario files to decrease their driveability, exposing
potential underlying issues in the \acrshort{ads} being tested in advance of it happening during
real world operation, avoiding causing severe damage to its operator and/or other external actors.

% TODO: Disse 2 sammendragene er veldig forskjellige. Er det OK?

\begin{otherlanguage}{norsk}
    \section*{Sammendrag}
    For å kunne få selvkjørende biler ut på veiene, må vi være sikre på at de er trygge. Trygge både
    for seg selv, sjåføren, og andre traffikanter. Men det ligger i en bils natur at den skal kunne
    brukes overalt, med alle mulige folk inne i bildet. Derfor er det teoretisk umulig å forutse
    alle mulige situasjoner og teste disse i forkant. På bakgrunn av dette fremmer vi i dette
    arbeidet en metode for å ta ibruk KI til å gjøre dagens testscenarioer \textit{mer utrygge} enn
    hva de allerede er. Å teste med disse forespeiles å ville kunne avdekke potensielle
    underliggende feil i bilens systemer, slik at de kan rettes før den volder skade ute i verden.
\end{otherlanguage}

\chapter*{Preface}
Here comes your preface, including acknowledgments and thanks.

The \LaTeX~sources for the project and their associated commit history is publically available on
the Github repo \url{https://github.com/orjahren/master}, along with an overview of various issues
and their progression.

The code used for the thesis experiments is available on the Github repo \url{https://github.com/orjahren/master-hefe}.

For better and worse, the thesis text has been written in its entirety without the use of generative
AI tools.