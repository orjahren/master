\chapter*{Abstract}

\acrfullpl{ads} should be capable of tackling any challenge they may be faced
with while driving. Before deploying such \acrshortpl{ads}, it is important to
be confident that they \emph{are} capable of handling potential challenges in a
fitting manner. But there is a near infinite set of potential scenarios that an
\acrshort{ads} may be faced with, making it impossible to predict and test on
all possible scenarios in advance. To this end, we propose using
\acrshortpl{llm} to decrease the driveability of our existing \acrshort{ads}
scenarios, enabling the \acrshort{ads} to face a more challenging test
environment. By using these more challenging test scenarios, we can either
\begin{inparaenum}
    \item cause it to fail and analyze why the \acrshort{ads} failed, or 
    \item increase our confidence in it being able to operate in challenging scenarios. 
\end{inparaenum}
We implement a tool --- \hefe{} --- for doing this and evaluate it with regard
to the jerk metric. We compare the jerk of the \acrshort{ads} in the `base' and
`enhanced' versions of the scenario, assessing if the driveability was decreased
and how the \acrshort{ads} responsed to this more challenging operating
environment. We also perform a literature review to survey the extant related
works and evaluate how \hefe{} fits into the state of the art. 
Experimental results show that using \acrshortpl{llm} to decrease
driveability is a promising strategy if the range of the changes the
\acrshort{llm} is allowed to make is limited, whereas problems related to
halluciantion and simulator crashes arise if the \acrshort{llm} makes excessive
changes to the original scenario. Based on the results, we present the research
and outline several strategies for improving \hefe{} in future research.


\begin{otherlanguage}{norsk}
    \chapter*{Sammendrag}
    Selvkjørende biler burde være i stand til å hanskes med enhver utfordring
    som måtte komme deres vei mens de er ute og kjører. Før man slipper
    selvkjørende biler fri ut i verden er det derfor viktig å være viss på at de
    \emph{er} i stand til å håndtere slike potensielle utfordringer. Men det er
    nært sagt et uendelig antall mulige scenarioer en bil kan komme til å stå
    ovenfor, slik at det er umulig å forutse alle scenarioer og teste på disse i
    forkant. Derfor foreslår vi å anvende store språkmodeller for å gjøre dagens
    eksisterende scenarioer mindre kjørbare enn hva de allerede er, slik at
    bilen får bryne seg på større utfordringer i forkant av å møte på dem ute i
    verden. Ved å benytte disse mer komplekse testscenarioene vil vi enten kunne
    \begin{inparaenum}
        \item trigge den til å feile slik at vi kan analysere hva som gikk galt
        og lære noe nytt om bilen, eller 
        \item være mer trygge på at bilen er i stand til å hanskes med komplekse
        scenarioer.
    \end{inparaenum}
    På bakgrunn av dette implementerer vi et verktøy --- \hefe{} --- for å gjøre
    nettopp dette og evaluerer det med hensyn på rykk-metriken. Vi sammenligner
    rykk i den selvkjørende bilen på tvers av baseformen av scenarioet og dets
    forbedrede versjoner, og ser på om kjørbarheten har blitt forverret fra
    hvordan den var opprinnelig og hvordan bilen forholder seg til dette mer
    utfordrende scenarioet. Vi har også gjennomført en litteraturgjennomgang og
    sett på et bredt utvalg forskning fra feltet slik at vi kan ta stilling til
    hvor \hefe{} passer inn i terrenget. Eksperimentene våre indikerer at dette
    konseptet med å bruke store språkmodeller til å senke kjørbarheten er en
    lovende fremgangsmåte all den tid språkmodellen begrenses i hvor
    \emph{brede} endringer den får lov til å gjøre. Dersom den slippes fri uten
    tøyler, støter vi på problemer knyttet til hallusinering og at simulatoren
    krasjer. Basert på resultatene presenterer vi selve forskningen, samt en
    rekke videre strategier for å forbedre \hefe{} i fremtidig arbeid.

\end{otherlanguage}

\chapter*{Preface}

Many thanks to my lovely supervisors Shaukat Ali and Karoline Nylænder, always encouraging me to
shoot for the stars and suggesting insightful ways to further the work with the research. Thanks to
Simula Research Laboratory and the Department of Informatics for enabling me to
do this fascinating work. Thanks to all my friends both within and outside of
\texttt{ifi} for motivating me to finish the thesis.

The \LaTeX~sources for the project and their associated commit history is publically available on
the GitHub repo \url{https://github.com/orjahren/master}, along with an overview of various issues
and their progression. Hopefully all the issues are resolved by the time you read this. The code
used for the thesis experiments is available on the GitHub repo
\url{https://github.com/orjahren/LLM4DD}. A similar issue tracking régime has been used for the
code and experiments. The project has been undertaken on Fedora Linux 42.

\begin{center}
    \vspace{2em}
    \begin{quote}
        \emph{``If I have seen further it is by standing on the shoulders of Giants.''}\\
        \vspace{1em}
        \textbf{-- Isaac Newton}
    \end{quote}
    \vspace{2em}
\end{center}
