\chapter{Further work}\label{chp:furtherWork}

\epigraph{Whenever a theory appears to you as the only possible one, take this as a sign that you have neither understood the theory nor the problem which it was intended to solve.}{Popper}

\section{LLM aspects}

\subsection{Different promtping strategies}

Overdrivelser? Typ "Det er veldig viktig for meg at du gjør dette fordi da blir
jeg glad"? Vise til litteratur som underbygger sånt.

\subsection{Temperature}
Hallucination.

\subsection{Pretraining?}

\subsection{Retrieval-augmented generation (RAG)}
Context, affordances.

\subsection{Model context protocol (MCP)}

Som RAG bare nyere. Bruke MCP som affordances for codegen.

\subsection{More models}

More models more good?

\subsection{Tool calling}

Can give the LLM access to tools, e.g. methods for adding objects etc.

\section{GUI visualisations}

Maybe: Frontend client - web GUI - Ivar
If Loki does its job effectively, we can create a web based frontend for doing the process. It could do the same as Loki, but with greater ease of use.
Having a GUI allows for making neat visualisations.
Motivate why our enhanced test cases are better by showing it.

\section{Instant validation of test case syntax}
Compiler-stuff. Syntax. Parsing.

% Og: Verifisere at spawn locations ikke overlapper/kolliderer

\section{Other datasets}

We used dataset x for our experiments. Scenario datasets y and z can also be used

\section{Impelementation oriented}

\subsection{The room for concurrency}

When evaluating \acrshort{ads} test cases, the test cases are independent of each
other. This means that our problem is \textit{embarrassingly parallelizable}
\footnote{\url{https://en.wikipedia.org/wiki/Embarrassingly_parallel}} and we can
trivially process several test cases in parallel. Due to practical limitations
in Carla, \textit{running} the test cases should however probably be done
sequentially. But \begin{inparaenum}
    \item prompting,
    \item enhancing, and
    \item validating,
\end{inparaenum}
can all be done concurrently. While Python lacks support of traditional threads,
it has some support for multiprocessing
\footnote{\url{https://docs.python.org/3/library/multiprocessing.html}}.

\subsection{Domain specific file format}

Ett felles filformat som holder all metadata om scenario og scenario prime, og
hvilken prompt som ble brukt osv