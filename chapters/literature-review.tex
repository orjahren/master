\chapter{Literature review}\label{chp:literatureReview}

\epigraph{The whole is greater than the sum of its parts.}{Aristotle}

TODO: Write literature review

Can move some things from related work such as LLM4AD?

\section{Graz University of Technology survey on LLM applications for ADSs} % TODO: Det er OK å referere til denne surveyen som dette?

\citeauthor{surveyLLMScenarioBasedTesting} give an extensive overview of some of the various ways
that \acrshortpl{llm} have been applied to scenario based testing of \acrlongpl{ads}.
The authors classify the various research efforts based on \begin{inparaenum}
    \item how they have employed the \acrshort{llm}, and
    \item to what end
\end{inparaenum}~\cite{surveyLLMScenarioBasedTesting}.
Their survey is continually updated, the last update having been made 2 months before the time of
writing\footnote{I.e. as of September 17th 2025, the last update to their
    \href{https://github.com/ftgTUGraz/LLM4ADSTest}{Github repo} was on July 23rd, 2025. The paper on
    Arxiv was last updated May 22nd 2025.}. This entails a certain overlap with some of the works we
review in \Nref{chp:relatedWork}.
% Not deterred by this, let us look at how they classify the works:

Not deterred by this, let us delve into the survey:
They start by highlighting the trend between the number of \acrshort{llm} surveys, and
\acrshort{ads} surveys -- while the trend was increasing from 2020-23, there was an explision in
\num{2024}, with about \num{200} works concering applying \acrshortpl{llm} for \acrlong{ads}
purposes being published~\cite[p. 1, figure (b)]{surveyLLMScenarioBasedTesting}. Furthermore, the
number of \acrshort{ads} studies has remained steady over the last \num{4}  years, wheras the number
of \acrshort{llm} studies has exploded in popularity~\cite[p. 1, figure
    (a)]{surveyLLMScenarioBasedTesting}. This indicates that a significant amount of the scientific
effort around \acrshortpl{ads} the last year, has been concerned with utilising \acrshortpl{llm}.
% TODO: Er det OK at jeg gjør utledninger som dette? (uten noe referanse)

\subsection{Meta survey review}

The article summarizes the field, pulling together various surveys of the
related subfields. Those being \begin{inparaenum}
    \item \acrshort{llm} surveys,
    \item surveys of scenario-based testing,
    \item general cases of \acrshortpl{llm} for \acrshortpl{ads}, and finally
    \item a broader review of surveys of \acrshortpl{llm} being applied for
    \textit{miscellaneous domains}
\end{inparaenum},
for each highlighting their specialized
foci~\cite[2]{surveyLLMScenarioBasedTesting}.
% TODO: Plural of "focus" is "foci", yes?

\subsection{The categories of ways of applying LLMs for ADS testing}

The authors posit that there are \num{0} major categories of works of
\acrshortpl{llm} being applied to \acrlongpl{ads}. They are.

\subsection{The \num{5} key challenges when applying LLMs for ADS testing}

Furthermore
