\chapter{Conclusion}\label{chp:conclusion}

\epigraph{This inductively justifies the conclusion that induction cannot justify any conclusions.}{David Deutsch}

In this master's thesis, we propose a tool -- \hefe{}~-- for using \acrlongpl{llm} to decrease the
driveability of \acrfull{ads} scenarios in order to expose underlying weaknesses in the
\acrshort{ads}. We show this work is in line with what is to be expected from other works in the
field, validating the concept for focusing on making less significant changes to the original
scenario with a focus on the jerk metric. When allowing the \acrshort{llm} to make excessive changes
to the scenario, problems related to hallucination and simulator crashes arise under our
\acrshort{llm}- and prompting strategy.

\section{Research question analysis}

Let us finish by returning to our initial \Nref{sec:RQs} and review to what
extent this work has answered them.

\subsection{\ref{rq:decrease-driveability}}

Recall that \ref{rq:decrease-driveability} says `\rqcontent{decrease-driveability}'.

\subsection{\ref{rq:no-human}}

Recall that \ref{rq:no-human} says `\rqcontent{no-human}'.


\section{Can large language models make our roads safer?}

While there also are other feasible strategies
for achieving this goal that fall outside of the scope of this
thesis\footnote{\acrshort{llm}-powered assistants may for example allow human drives to maintain
    their focus on the road.}, when focusing on exposing possible faults in \acrshortpl{ads} thanks to
decreased driveability in simulator scenarios, we have shown the answer to be `yes'.