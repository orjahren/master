\chapter{Conclusion}\label{chp:conclusion}

\epigraph{This inductively justifies the conclusion that induction cannot justify any conclusions.}{David Deutsch}

In this master's thesis, we propose a tool -- \hefe{}~-- for using
\acrshortpl{llm} to decrease the driveability of \acrshort{ads} scenarios in
order to increase confidence in the \acrshort{ads} and expose underlying
weaknesses in the system. We do a literaview with both a theoretical and applied
focus. We show that this work is in line with what is to be
expected from comparison with other related works in the field, validating that
the \hefe{} concept works quite well for focusing on making less significant changes to
the original scenario with a focus on the jerk metric. When allowing the
\acrshort{llm} to make excessive changes to the scenario, the results indicate
that problems related to hallucination and simulator crashes arise under our
current \acrshort{llm}- and prompting strategies.

The \hefe{} tool is a modularized pipeline tool, consisting of the components
\texttt{Odin}, \texttt{Thor}, and \texttt{Loki}, each component being
respectively responsible for handling \begin{inparaenum}
    \item \acrshort{llm} integration,
    \item \acrshort{ads} integration, and 
    \item user-based orchestration.
\end{inparaenum}

We used the \acrshortpl{llm} \texttt{Mistral 7.2}B and \texttt{gemini-2.5-flash}
and various original prompts. The scenarios we used for evaluating the tool came
from the example set provided with the Carla scenario runner. We primarily
focused on the scenarios \texttt{Accident}, \texttt{CutIn},
\texttt{NoSignalJunctionCrossing} and \texttt{FollowLeadingVehicle}. The results
obtained indicate a \num{63}\% error rate of running a scenario after it has
been enhanced. We classified these errors into \num{4} categories based on what
caused the execution to fail.

\newpage

We propose several strategies for improving this initial version of the tool in
\Nref{chp:furtherWork}.

In conclusion, this thesis shows that \acrshortpl{llm} can indeed be used to
decrease the driveability of \acrshort{ads} scenarios. The
\acrshort{llm}-based approach presented in this thesis and evaluated by
implementing the \hefe{} tool, appears as a promising contribution to
furthering the state of the art of \acrshort{ads} verification research.