\chapter{Results}\label{sec:results}

\epigraph{I have not failed. I've just found 10,000 ways that won't work.}{\textit{Thomas A. Edison}}

Our results show that the initially proposed solution of feeding bare
\acrshort{ads} scenarios represented by Python code into \acrshortpl{llm}, does
not yield any meaningful results. This is caused by various reaons. The
following discusses \begin{inparaenum}
    \item why this is, and 
    \item ways by which it can be remedied in future work \end{inparaenum}.

\section{Output of the \acrshort{llm}}

Depending on the prompt, our results show that it \emph{is} possible to get
reasonable-looking Python out of the \acrshort{llm}. One somewhat annoying
detail is their bent to mark the code as specific syntax, applying a
Markdown-formatted code block indicating both that the output \emph{is} code,
and what language it is in.,to the first and last line of the output (Listing \ref{lst:llmOutputMarkdown}). 
% \begin{lstlisting}[language=Markdown]
\begin{lstlisting}[caption={\acrshort{llm}-generated Python code with Markdown syntax. The bracketed part on line 3 has been added for demonstration purposes, removing the actual code for brevity.}, label={lst:llmOutputMarkdown}]
```python

[ scenario code ]

```
\end{lstlisting}

Upon manually removing these syntactic artefacts, we can go ahead with executing
the scenario. But as previously mentioned, we are unable get any meaningful
results. This comes down to \begin{inparaenum}
    \item halluciantion of Python code, and 
    \item Carla problems
\end{inparaenum}. 
Writing code to programatically these lines is naturally trivial, but we have
not gone ahead with implementing this due to having the focus being on resolving
the other issues that prevented the scenarios from being executed properly.

Something worth noting is that the \acrshort{llm} demonstrates a promising
ability to explain back to the user \emph{how} it ehnahced the scenario, e.g. in
the fom of bullets in a docstring of the output code (see listing \ref{lst:llmOutputExplenation}).

% NOTE: Dette eksempelet er hentet fra CiE-6.
\begin{lstlisting}[caption={Head of an \acrshort{llm}-enhanced scenario, highlighting how the \acrshort{llm} can add an explenation of how it enhanced the scenario.}, label={lst:llmOutputExplenation}, language={Python}]
#!/usr/bin/env python

# Copyright (c) 2019-2020 Intel Corporation
#
# This work is licensed under the terms of the MIT license.
# For a copy, see <https://opensource.org/licenses/MIT>.

"""
Cut in scenario:

The scenario realizes a driving behavior on the highway.
The user-controlled ego vehicle is driving straight and keeping its velocity at a constant level.
Another car is cutting just in front, coming from left or right lane.

The ego vehicle may need to brake to avoid a collision.

Enhanced scenario:
- Increased background traffic with varying speeds to create a more crowded environment.
- Challenging weather conditions (heavy rain, fog, strong winds) to reduce visibility and grip.
- Nighttime setting to further decrease visibility.
- Randomization of speeds and trigger distances for increased unpredictability.
"""
[...]
\end{lstlisting}

\subsection{Hallucinations in the enhanced scenarios}

The \acrshort{llm} typically seems to be on the right track, outlining something
that \emph{sounds} like a good approach to satisfying our prompt of decreasing
the driveability of the scenario. But in practice, it will often hallucinate
methods that don't exist, or use terms and phrasing that are not valid keywords
in the Carla specificication. This is in line with what was found by e.g.
\citeauthor{autoSceneGen}~\cite[14542]{autoSceneGen} (See \Nref{sec:autoSceneGen} in
\Nref*{sec:relatedWork}).

% TODO: Legge til ekesempler her
% TODO: Er vi sikre på at dette handler om spesifikt hallusinering og ikke noe annet?
\subsubsection{Non-existing methods}

As mentioned, the \acrshort{llm} seems to have the right idea of what it can do
to achieve the stated goal. But the way that it goes about obtaining it, does
not always work. The enhanced scenario code will often call methods that don't
exist. This leads to a runtime exception in the scenario runner when executing
the enhanced scenario. 

% TODO: Legge til ekesempler her
% TODO: Er vi sikre på at dette handler om spesifikt hallusinering og ikke noe annet?
\subsubsection{Non-existing arguments}

In a similar vein to the non-existing methods, non-exisiting \emph{arguments}
were also shown to appear. The \acrshort{llm} could simply call methods that
were already being used, with additional arguments that made semeantic sense,
but that were not a part of the function definition. This also causes runtime
exceptions in the scenario runner.

% TODO: Burde refereree / kildeføre / vise til noe forankring for disse keyword-eksemplene.
\subsubsection{Illegal property keywords}

Another trend we observed was the usage of various keywords that simply don't
exist in the Carla repetoire. Where Carla would recognize the word `snowstorm',
the \acrfull{llm} proposed using the word `blizzard'.


\subsection{Carla crashes with certain scenarios}

There appears to be a bug in Carla version 0.9.15\footnote{Which is the version
employed for this project.} which causes the program to \emph{hard crash} when
executing certain scenarios with metric recording enabled. This has been
reported to the project Github\footnote{By several members of the scientific community, see e.g.
\begin{itemize}\item  \url{https://github.com/carla-simulator/carla/issues/9170} and \item \url{https://github.com/carla-simulator/carla/issues/9152}\end{itemize}}, but as of 2025-09-30 it has not been resolved.
Testing shows that the same scenarios may be ran without crashing when
\textbf{not recording}, but this naturally has severe implications for our
opportunities of obtaining data from the simulation run. The `record' function
of the scenario runner is the crux of measuring the driveability of the
scenario. 


\section{Metrics used for evaluation}

We measure several metrics for evaluating the driveability of the scenario. The
principal is \emph{jerk}.

Due to the above resons with getting the enhanced scenarios to run, there is
however minimal data to bases any qualitative analysis on.