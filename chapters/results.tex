\chapter{Results}\label{chp:results}

\epigraph{It doesn't matter how beautiful your theory is, it doesn't matter how smart you are. If it doesn't agree with experiment, it's wrong.}{Feynman}

This chapter will survey a selection of results from performing various experiments (see
\Nref{chp:experiments}). This chapter will \emph{present} the results, and they will then be
analysed later in \Nref{sec:resultAnalysis}.

See the listings in the \Nref{sec:fileDiffs} appendix for some demonstrations of what the
\acrlong{llm} is capable of doing to an scenarios with regard to the syntax of scnearios. Going
ahead, we first look at some general aspects that are shared between all our experiments, before
narrowing the scope and reviewing a selection of individual scenarios, highlighting the value added
by the \hefe~tool.

Table \ref{tab:scenarioFailures} renders a table of the status of executing various
scenarios after they have been processed by a Gemini \acrshort{llm}.

\begin{table}[htbp]
    \centering
    \caption{Statistics of simulator execution of LLM-enhanced scenarios, accross all prompts}
    \begin{tabular}{l c}
        \toprule
        Execution status             & Number of Scenarios \\
        \midrule
        Unexpected keyword argument  & 8                   \\
        Hard simulator crash         & 3                   \\
        Illegal object placement     & 1                   \\
        Non-existing import          & 3                   \\
        \midrule
        No execution issues          & 9                   \\
        \midrule
        Number of enhanced scenarios & 24                  \\
        Number of crashed scenarios  & 15                  \\
        Failure ratio                & 63\%                \\
        \bottomrule
    \end{tabular}
    \label{tab:scenarioFailures}
\end{table}

% \subsection*{Carla crashes with certain scenarios}\label{sec:resCarlaUnstable}

There appears to be a bug in Carla version 0.9.15\footnote{Which is the version
    employed for this project.} which causes the program to \emph{hard crash} when
executing certain scenarios with metric recording enabled. This has been
reported to the project GitHub\footnote{By several members of the scientific community, see e.g.
    \begin{itemize}\item  \url{https://github.com/carla-simulator/carla/issues/9170}, \item
              \url{https://github.com/carla-simulator/carla/issues/9152} and \item
              \url{https://github.com/carla-simulator/carla/issues/9349}\end{itemize}}, but as of 2025-10-15
it has not been resolved. Testing shows that the same scenarios may be ran without crashing when
\textbf{not recording}, but this naturally has severe implications for our
opportunities of obtaining data from the simulation run. The `record' functionality
of the scenario runner is the crux of measuring the driveability of the
scenario.

Note that this is a different kind of problem from those presented in
\Nref{sec:resultsHallucinations} -- \emph{this} problem is relevant for \emph{base} scenarios,
provided first party by the Carla simulator team. A major consequence of this is that it hindered
what sort of base scenarios we could experiment with. We were naturally unable to experiment with
enhancing base scenarios that we were unable to run, as we would have no baseline to measure
against, and it is highly improbable that the scenario would magically start working after having
gone through an \acrshort{llm} with a prompt aiming at \emph{worsening} its complexity.

\section{Examples of enhanced scenarios}\label{sec:examplesOfEnhancedScenarios}

With all these generalities in mind, let us now narrow the scope and evaluate some tangible scenarios. We will contrast the baseline, original, scenarios, with
some that have been enhanced by a \acrshort{llm}, focusing on what changes the \acrshort{llm}
proposes and how they affect the driveability of the scenario.

The base scenarios used for these experiments come from the official Carla scenario runner software
library\footnote{\url{https://github.com/carla-simulator/scenario_runner}}, but the concept is
applicable to scenarios of other repositories as well. Several alternative options are presented in
\Nref{sec:relatedWork}. Due to the aforementioned challenges with getting scenarios to run on the
Carla simulator, these basic scenarios are used for the purposes of the thesis experiments, serving
as a validation of the concept and laying the groundwork for adapting the method to other scenario
sources in the future.

As mentioned in \cref{sec:testCaseRunner}, Carla scenario excutions are saved to binary log files.
But these files are \emph{huge}, typically being several hundred megabytes depending on the duration
of the scenario execution\footnote{Keep in mind that they record data for all actors in the scenario
    over time.}. As such, publishing all our raw files is not feasible. A selection is available in the
\href{https://github.com/orjahren/master-hefe/tree/main/odin/experiments/results}{\hefe~Github
    repo}.

\subsection{Base scenario: Follow vehicle}\label{sec:followVehicleResults}

The `follow vehicle' is the most basic kind of scenario out there. It simply consists of one ego
vehicle, and one external vehicle. Our ego is to follow the other vehicle along a straight road in a
residential area.

\begin{figure}[htbp]
    \centering
    \includegraphics[width=\textwidth]{experiment-material/follow-base-startpoint.png}
    \caption{A screenshot of the base `follow' scenario where our ego chases an external actor.}\label{fig:followBaseStartpoint}
\end{figure}

\Cref{fig:followBaseStartpoint} gives a visual representation of the initial state of the scenario.
Due to the low complexity of this scenario, we won't gain any substantial insight into how well our
\acrshort{ads} works, if it is to execute the scenario properly. We want to make it more complex in
order to provide our \acrshort{ads} with a more challenging environment in which it is more likely
to fail.

To this end, we employ an \acrshort{llm} to decrease the driveability.

If we prompt the \acrshort{llm} with a broad allowance of ways of decreasing the driveability such
as the prompt demonstrated in listing \ref{lst:basicPrompt}, we run into issues with excessive
hallucination. The \acrshort{llm} wants to import a class that simply does not exist in the Carla
\acrshort{api}. See listing \ref{lst:hallucinatedFollowError} in the \Nref{sec:errorMessages} appendix for the complete error message.


\begin{lstlisting}[language=python, label={lst:basicPrompt}, caption={The most basic prompt first used in the experiments. This leads to excessive halluciantion.}]
            "no_explanation": lambda python_carla_scenario_raw: f"""
    1 - Context: You are a tool for decreasing the driveability of scenarios in the driving simulator Carla.
    2 - Task: Decrease the driveability of the scenario by enhancing it with more details and complexity.
    3 - Input, the Python specification for the scenario: {python_carla_scenario_raw}
    4 - Output: An enhanced version of the scenario with additional details and
    complexity, still in Python carla scenario format. Only ever output the code,
    without any additional text or explanation.
    """,
\end{lstlisting}


Due to not being able to run, there is not much to show for here. We need to revise the prompt and
discourage such hallucinations in order to obtain meaningful results. In line with the
\Nref{chp:experiments}, we iterate on the prompt. We first tell the \acrshort{llm} to
\emph{strictly adhere to the Carla \acrshort{api}}.

The complete output resulting from this prompt is shown in listing \ref{lst:hallucinatedFollowDiff}
in the appendix.
Keep in mind that \acrshortpl{llm} by nature are not deterministic, and as such it is probable
that trying to reproduce this output might not be straight-forward.

\begin{lstlisting}[language=python, label={lst:strictPrompt}, caption={A slightly more advanced prompt instructing the \acrshort{llm} to strictly adhere to the Carla \acrshort{api}.}]
        "no_explanation_strict": lambda python_carla_scenario_raw: f"""
    1 - Context: You are a tool for decreasing the driveability of scenarios in the driving simulator Carla.
    2 - Task: Decrease the driveability of the scenario by enhancing it with more details and complexity.
    3 - Input, the Python specification for the scenario: {python_carla_scenario_raw}
    4 - Output: An enhanced version of the scenario with additional details and
    complexity, still in Python carla scenario format. Only ever output the code,
    without any additional text or explanation. It is important that you only
    use methods and classes that are part of the official Carla API, and do not
    invent new ones or use non-existent ones.
    """,
\end{lstlisting}

As shown in listing \ref{lst:strictPrompt}, we iterate by instructing the \acrshort{llm} to make
sure to strictly adhere to the Carla \acrshort{api}. This yields a similar problem where the
\acrshort{llm} attempts to make an import that does not exist. This diff is presented in listing
\ref{lst:hallucinatedFollowDiffStrict}.

\newpage % Legger inn ny side for å få hele listingen på én side.

Iterating further, we realize that we must walk before we can run. We therefore instruct the
\acrshort{llm} to make as few changes as possible. The intuition being that if it does this and
relies on the options that are already present in the file, it is more plausible that we will get a
runnable output. This prompt is presented in listing \ref{lst:minimalChangesPrompt}.

\begin{lstlisting}[language=python, label={lst:minimalChangesPrompt}, caption={A prompt instructing the \acrshort{llm} to make as few changes as possible to increase the likelyhood of it working without issues.}]
        "minimal_changes": lambda python_carla_scenario_raw: f"""
    1 - Context: You are a tool for decreasing the driveability of scenarios in the driving simulator Carla.
    2 - Task: Decrease the driveability of the scenario by enhancing it with
    more details and complexity, using only methods that are part of the
    official Carla API, version 0.9.15.
    3 - Input, the Python specification for the scenario: {python_carla_scenario_raw}
    4 - Reasoning: Think step by step about how to make the scenario more complex and less driveable, considering possible obstacles, traffic, weather, and other factors using only the official Carla API.
    5 - Output: Only output the enhanced scenario code in Python Carla scenario
    format, with no additional text or explanation. Make sure to only use
    methods and concepts that are already present in the input scenario, and
    do not introduce any new methods or concepts. The changes should be as
    minimal as possible while still achieving the goal of decreasing driveability.
    """,
\end{lstlisting}

This approach works well. We have been able to obtain several working scenarios with decreased
driveability with this prompting strategy.

Let us now review some of these enhanced versions of the scenario.

\subsubsection{Enhanced scenarios}

The enhanced scenarios are simply presented here, and then analysed later in \Nref{sec:resultAnalysis}.

% Disse 2 er hhv mod-1 og mod-2 fra Hefe

\begin{figure}[htb]
    \centering
    \includegraphics[width=\textwidth]{experiment-material/follow-minimally-enhanced-1-startpoint.png}
    \caption{A screenshot of a minimally enhanced `follow' scenario with a truck in the road.}\label{fig:followMinimallyEnhanced1StartPoint}
\end{figure}

\Cref{fig:followMinimallyEnhanced1StartPoint} gives a visual representation of the initial state of
one enhanced scenario.

Another result (\Cref{fig:followMinimallyEnhanced2StartPoint}) places a vehicle parked on the edge
of the road. This is in line with our prompt, representing a change that is both \begin{inparaenum}
    \item minimal, and still
    \item decreasing driveability.
\end{inparaenum}

\begin{figure}[htb]
    \centering
    \includegraphics[width=\textwidth]{experiment-material/follow-minimally-enhanced-2-startpoint.png}
    \caption{A screenshot of another minimally enhanced `follow' scenario with a van parked on the
        side of the road.}\label{fig:followMinimallyEnhanced2StartPoint}
\end{figure}

% TODO: Også legge inn den tredje?

Let us now review the jerk metric for these variations of the scenario (\Cref{fig:jerkComparison}).

\begin{figure}[!htb]
    \centering
    \subfloat[Jerk in the base scenario, rendered in fig~\ref{fig:followBaseStartpoint}\label{fig:followBaseJerk}]{
        \includegraphics[width=0.45\textwidth]{experiment-material/follow-base-jerk.png}
    }
    \hfill
    \subfloat[Jerk of the ego vehicle in the first minimally enhanced `follow'
        scenario, the one from fig~\ref{fig:followMinimallyEnhanced1StartPoint}\label{fig:followMinimallyEnhanced1Jerk}]{
        \includegraphics[width=0.45\textwidth]{experiment-material/follow-minimally-enhanced-1-jerk.png}
    }
    \hfill
    \subfloat[Jerk of the ego vehicle in the 2nd minimally enhanced `follow' scenario, the one from
        fig~\ref{fig:followMinimallyEnhanced2StartPoint}\label{fig:followMinimallyEnhanced2Jerk}]{
        \includegraphics[width=0.45\textwidth]{experiment-material/follow-minimally-enhanced-2-jerk.png}
    }
    \caption{Jerk of the ego vehicle in the base and enhanced `follow' scenarios.}
    \label{fig:jerkComparison}
\end{figure}

\FloatBarrier{} % gjør så alle floats (bilder) MÅ inn i PDFen før man kan rendre
% mer tekst


\subsection{Base scenario: Accident}\label{chp:resultsAccidentScenario}

The accident scenario is a bit more complex than the `follow' scenario, representing a scene on a
highway where several cars have piled up in front, and the ego vehicle comes around a corner.
\Cref{fig:accidentBaseStartPoint} visually renders the starting point of the scenario. Note the
`accident ahead' sign on the right-hand side of the road.
Upon continuing further, a pileup of several vehicles appear in the distance.
The base scenario ends with our ego coming to a halt behind the piled up vehicles.
\Cref{fig:accidentBaseProgression} shows the start point and the progression of the ego continuing
around the corner.
\Cref{fig:accidentBaseStoppedVisual} shows how the situation ends -- with the \acrshort{ads} ego
stopping behind the pileup of other vehicles.

\begin{figure}[h]
    \centering
    \subfloat[Start of the `accident' scenario.\label{fig:accidentBaseStartPoint}]{
        \includegraphics[width=0.48\textwidth]{experiment-material/accident-pics/base/startpoint.png}
    }
    \subfloat[Ego vehicle underway.\label{fig:accidentBaseUnderway}]{
        \includegraphics[width=0.48\textwidth]{experiment-material/accident-pics/base/underway.png}
    }
    \caption{Progression of the base `accident' scenario: start and underway.}
    \label{fig:accidentBaseProgression}
\end{figure}


\begin{figure}[h]
    \centering
    \subfloat[The ego stopped behind the pileup.\label{fig:accidentBasePileupBack}]{
        \includegraphics[width=0.48\textwidth]{experiment-material/accident-pics/base/stopped.png}
    }
    \subfloat[The same situation from a different angle.\label{fig:accidentBasePileupFreecam}]{
        \includegraphics[width=0.48\textwidth]{experiment-material/accident-pics/base/side-view.png}
    }
    \caption{The final ego vehicle state in the base `accident' scenario.}
    \label{fig:accidentBaseStoppedVisual}
\end{figure}

As for the `follow' scenario, we will enhance the scenario using \acrshortpl{llm} and measure the
jerk of the ego vehicle between the executions of the scenarios, only presenting the enhanced
scenarios and their data points here, and then analyzing them later in \Nref{sec:resultAnalysis}.

\subsubsection{Enhanced scenarios}

The first enhancement is done using the `minimal changes'-prompt (listing
\ref{lst:minimalChangesPrompt}). It is rendered visually in \Cref{fig:accidentMod1StoppedVisual} and
the complete diff of this enhancement is rendered in listing \ref{lst:appendixDiffAccidentMod1} in
\Cref{sec:appendixAccidentDiffs}.

\begin{figure}[h]
    \centering
    \subfloat[The ego stopped behind the pileup, \\with the vehicles on bollards.\label{fig:accidentMod1PileupBack}]{
        \includegraphics[width=0.48\textwidth]{experiment-material/accident-pics/mod-1/back.png}
    }
    \subfloat[The same situation from a different perspective.\label{fig:accidentMod1PileupFreecam}]{
        \includegraphics[width=0.48\textwidth]{experiment-material/accident-pics/mod-1/freecam.png}
    }
    \caption{The final ego vehicle state in the minimally enhanced `accident' scenario.}
    \label{fig:accidentMod1StoppedVisual}
\end{figure}


\begin{lstlisting}[language=python, label={lst:jerkPrompt}, caption={A prompt instructing the \acrshort{llm} to make as few changes as possible, while maximizing a specific metric.}]
  "minimal_changes_specific_metric": lambda python_carla_scenario_raw, specific_metric: f"""
    1 - Context: You are a tool for decreasing the driveability of scenarios in the driving simulator Carla.
    2 - Task: Decrease the driveability of the scenario by enhancing it with
    more details and complexity, using only methods that are part of the
    official Carla API, version 0.9.15.
    3 - Input, the Python specification for the scenario: {python_carla_scenario_raw}
    4 - Reasoning: Think step by step about how to make the scenario more complex and less driveable, considering possible obstacles, traffic, weather, and other factors using only the official Carla API.
    5 - Output: Only output the enhanced scenario code in Python Carla scenario
    format, with no additional text or explanation. Make sure to only use
    methods and concepts that are already present in the input scenario, and
    do not introduce any new methods or concepts. The changes should be as
    minimal as possible while still achieving the goal of decreasing
    driveability.
    Focus on making the scenario more difficult with respect to the
    following specific metric: {specific_metric}
    """,
\end{lstlisting}

In order to obtain more interesting results, we again iterate on the prompt and add the requirement
of the \acrshort{llm} optimizing for the \emph{jerk} metric. The iterated prompt is rendered in
listing \ref{lst:jerkPrompt}. Note how the prompt is generic and takes the metric as an argument,
allowing for other metrics to be used in a similar fashion.



So, what \emph{did} the \acrshort{llm} do? For brevity, the diff is rendered in the appendix
\Cref{sec:appendixAccidentDiffs} (listing \ref{lst:jerkOptimizedAccidentMod2}). From inspecting the
diff, it is clear that the \acrshort{llm} has focused on \emph{tweaking} the already existing
properties of the scenario, opting to not add additional ontological entities. Not only that, but it
has also carried out the modifications for \emph{other} scenarios that are also represented in the
same file. Thus, for our `accident` scenario, it has only really done the following enhancement, as
seen in listing \ref{lst:jerkDiff}.

\begin{lstlisting}[language=diff, label={lst:jerkDiff}, caption={The relevant subset of the diff from instructing the \acrshort{llm} to make as few changes as possible, while maximizing a specific metric.}]
    @@ -69,9 +69,9 @@ class Accident(BasicScenario):
         self._first_distance = 10
         self._second_distance = 6
 
-        self._trigger_distance = 50
+        self._trigger_distance = 20  # Decreased to force sharper reactions
         self._end_distance = 50
-        self._wait_duration = 5
+        self._wait_duration = 1  # Decreased to allow less reaction time
         self._offset = 0.6
\end{lstlisting}

Therefore, we do yet another iteration of the prompt, underlining what scenario the \acrshort{llm}
should focus on (listing \ref{lst:jerkPromptSpecificScenario}).

\begin{lstlisting}[language=python, label={lst:jerkPromptSpecificScenario}, caption={A prompt instructing the \acrshort{llm} to make as few changes as possible, while maximizing a specific metric only in a specified scenario.}]
    "minimal_changes_shared_file_specific_metric": lambda python_carla_scenario_raw, scenario_name, specific_metric: f"""
    1 - Context: You are a tool for decreasing the driveability of scenarios in the driving simulator Carla.
    2 - Task: Decrease the driveability of the scenario by enhancing it with
    more details and complexity, using only methods that are part of the
    official Carla API, version 0.9.15.
    3 - Input, the Python specification for the scenario:
    {python_carla_scenario_raw}. Note that there are several scenarios in the file,
    but you should only modify the one called {scenario_name}. Don't change any of the
    other scenarios.
    4 - Reasoning: Think step by step about how to make the scenario more complex and less driveable, considering possible obstacles, traffic, weather, and other factors using only the official Carla API.
    5 - Output: Only output the enhanced scenario code in Python Carla scenario
    format, with no additional text or explanation. Make sure to only use
    methods and concepts that are already present in the input scenario, and
    do not introduce any new methods or concepts. The changes should be as
    minimal as possible while still achieving the goal of decreasing
    driveability.
    Focus on making the scenario more difficult with respect to the
    following specific metric: {specific_metric}
    """,
\end{lstlisting}


\begin{figure}[h]
    \centering
    \subfloat[A screenshot from the jerk optimized \\`accident' scenario with the focused on\\ modifying the correct scenario. See
        listing \ref{lst:jerkOptimizedAccidentMod3}.\label{fig:accidentMod3FinalFreecam}]{
        \includegraphics[width=0.48\textwidth]{experiment-material/accident-pics/mod-3/freecam.png}
    }
    \subfloat[A screenshot from the jerk optimized `accident' scenario with the
    ego stopped.\label{fig:accidentMod2FinalFreecam}]{
        \includegraphics[width=0.48\textwidth]{experiment-material/accident-pics/mod-2/freecam.png}
    }
    \caption{The final ego vehicle state in two enhanced `accident' scenarios.}
\end{figure}

Note how the vehicle behind the taxicab is moving in \Cref{fig:accidentMod3FinalFreecam}.

\begin{figure}[htb]
    \centering
    \subfloat[Jerk in the base scenario, rendered in
        fig~\ref{fig:accidentBaseProgression}\label{fig:accidentBaseJerk}]{
        \includegraphics[width=0.45\textwidth]{experiment-material/accident-pics/base/jerk.png}
    }
    \hfill
    \subfloat[Jerk of the ego vehicle in the first minimally enhanced `accident' scenario, the one
        from fig~\ref{fig:accidentMod1StoppedVisual}\label{fig:accidentMod1Jerk}]{
        \includegraphics[width=0.45\textwidth]{experiment-material/accident-pics/mod-1/jerk.png}
    }
    \hfill
    \subfloat[Jerk of the ego vehicle in the 3rd minimally enhanced `accident' scenario, the one
        from fig~\ref{fig:accidentMod3FinalFreecam}\label{fig:accidentMod3Jerk}]{
        \includegraphics[width=0.45\textwidth]{experiment-material/accident-pics/mod-3/jerk.png}
    }
    \caption{Jerk of the ego vehicle in the base and enhanced `accident' scenarios.}
    \label{fig:jerkComparisonAccident}
\end{figure}
