\chapter{Introduction}\label{chp:introduction}

\epigraph{A problem well stated is a problem half solved.}{\textit{Charles F. Kettering}}

\section{Motivation}

\textbf{Conventional cars} are ubiquitous in society. Whether for freight trafficking or for humans, cars have great flexibility with their ability to go wherever without requiring tailored infrastructure
such as railway tracks. They do, however, have one major weak point --- the human driver. For this
reason, industry and academia have put forward efforts to enhancing cars with \acrfull{ads}
capabilities.
By \textbf{empowering humans} with autonomous vehicles, it is expected that traffic efficiency will
increase and road fatalities will fall.

\textbf{Due to the critical safety situation of manoeuvring a car} In a public setting where other external
actors are present, it is essential that \acrlongpl{ads} are thoroughly tested before they are
deployed so that they are confirmed to be sufficiently safe and capable of handling the situations in which
they may typically end up.
But due to the complicated nature of the typical \acrshort{ads} operating environment, coming up with
exhaustive system test solutions is near impossible.
For this reason we want a way of testing the system that is capable of pushing the \acrlong{ads} to
its limits such that we can measure its performance and see if it is capable of
handling complex scenarios.

\textbf{Having an existing repository of \acrlong{ads} test cases,} such as
DeepScenario we wish to improve them. \textbf{\acrfullpl{llm}} have demonstrated
great capabilities of context learning and emergent abilities, which begs
the question of their  applicability for \acrshort{ads} testing.  There are
various methods of testing  \acrlong{ads}. Can these existing test methods be
improved by applying \acrshort{llm} technology to them?

% Yet, testing is \textit{important} for \acrlong{ads}, and text case generation is costly. We
% therefore pose the question: Can \acrshort{llms} be applied for (1) lowering the
% cost of testing, and (2) increasing the thouroughness of \acrlong{ads} testing?

\section{Problem description}\label{sec:problemDescription}

% TODO: Denne kan renames til noe a la "human factors"?
\subsection{Cost and bias}

Traditional techniques for obtaining \acrshort{ads} scenarios rely on high skilled manual labour.
This incurs a significant cost, and is a major limitation in obtaining a large number of good
scenarios, free from the bias of the author
% TODO: Bias er ikke relevnt her??


\subsection{Impossible to test all scenarios}
Furthermore, even if we were to imagine a world in which we had infinite \begin{inparaenum}
    \item time and
    \item money
\end{inparaenum}, we would not be able to successfully account for every possible scenario. This is
a reality we need to deal with. One possible measure of remedying with this, could be to
\textit{decrease} the driveability of our existing scenarios. Decreasing the driveability is not the
same as suddenly having access to the infinite set of possible scenarios, but it is reasonable to
infer that begin able to \textit{test} the \acrshort{ads} (in a simulator) on these enhanced
low-driveability scenarios will leave it better fit for encountering other low-driveability
scenarios in the wild during operation.

\subsection{Edge cases}

Edge cases can be a major issue for \acrshort{ads} adoptation. The \textit{tail problem} as it is
known in the \acrshort{ml} field posits that \acrshort{ml} tasks are faced with a long tail of
unseen cases. We can map these unseen cases, to our unseen \acrshort{ads} scenarios. Because of
this, an \acrshort{ads} can be at risk of encountering an unseen edge case scenario during
operation -- something for which it might never have been tested.
% TODO: Burde referere ting om tail problem? 
Arguing that the \acrshort{ads} would probably crash simply due to it finding itself in an unseen
scenario is not logical. But it is important to keep in mind that the end we are pursing in the
broader adaption of \acrfullpl{ads}, is increased saftey and efficiency on our roads. Not
sufficiently testing the \acrshort{ads} before deploying it would not serve our goal of increasing
road safety -- it would be a gamble with human lives.

\section{Research questions}\label{sec:RQs}

Based on the \Nref{sec:problemDescription}, these are the formal research
questions:

% Define each RQ as a macro
\newcommand{\rqdecreasedriveability}{Can \acrlongpl{llm} be used to decrease the driveability of \acrlong{ads} simulator scenarios?}
\newcommand{\rqnohuman}{Is it feasible to employ \acrshortpl{llm} for obtaining unseen scenarios for \acrshort{ads} testing without human intervention?}

\begin{questions}
    \item \rqdecreasedriveability
    \defrq{decrease-driveability}{\rqdecreasedriveability}\label{rq:decrease-driveability}
    \item \rqnohuman
    \defrq{no-human}{\rqnohuman}\label{rq:no-human}
\end{questions}


\section{Thesis overview}

The thesis is structured as follows:

\begin{itemize}
    \item \Cref{chp:introduction} defines the specific problem addressed in this work.
    \item \Cref{chp:background} introduces key concepts related to \acrshortpl{ads} and \acrshortpl{llm}.
    \item \Cref{chp:literatureReview} reviews the current state of research in the field.
    \item \Cref{chp:relatedWork} discusses related applied works and lessons learned.
    \item \Cref{chp:solutionProposal} details our proposed solution and its technical aspects.
    \item \Cref{chp:experiments} describes the experimental setup used to evaluate the solution.
    \item \Cref{chp:results} presents the findings from the experiments.
    \item \Cref{chp:discussion} analyzes and compares the results to existing work.
    \item \Cref{chp:furtherWork} suggests directions for future research.
    \item \Cref{chp:conclusion} summarizes the main contributions and findings.
\end{itemize}

Two appendices are included: \Nref{sec:fileDiffs} and \Nref{sec:errorMessages}.
