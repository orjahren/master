\chapter{Introduction}\label{chp:introduction}

\epigraph{A problem well stated is a problem half solved.}{Charles F. Kettering}

This chapter presents the motivation and problem statement of the thesis,
condensing the the problem statement to a set of formalized research questions.
Finally, we present the structure of the thesis with an outline of the topics of
each chapter.

\section{Motivation}

Conventional cars are ubiquitous in society~\cite[1]{thorgersen2021why}. Whether for freight
trafficking or for humans, cars have great flexibility with their ability to go wherever without
requiring tailored infrastructure such as railway tracks. They do, however, have one major weak
point --- the human driver~\cite[67]{Riener2010}. For this reason, industry and academia have put
forward efforts to enhancing cars with \acrfull{ads} capabilities. By empowering humans with
autonomous vehicles, it is expected that traffic safety and efficiency will increase along with
comfort as well as enabling the development of several other new transportation methods~\cite[1-2,
    1]{Eichberger2020, weiADSAdoption}.

Due to the critical safety situation of operating a car, it is essential that \acrshort{ads} are
thoroughly tested before they are deployed so that they are verified to be sufficiently safe and
capable of handling the situations in which they may typically end
up~\cite[1]{surveyLLMScenarioBasedTesting}.
But due to the complicated nature of the typical
\acrshort{ads} operating environment, coming up with exhaustive system test solutions is near
impossible~\cite[52]{DeepScenario}. For this reason, we want a way of testing the system that is
capable of pushing the \acrshort{ads} to its limits such that we can measure its performance and see
if it is capable of handling critical scenarios~\cite{yao2025agentsllm}.

Existing methods for testing \acrshortpl{ads} typically rely on driving billions of miles, but this
incurs high cost, and is time-intensive~\cite[1]{surveyLLMScenarioBasedTesting}. To address some
of these concerns, \acrshort{ads} simulators have been utilised. But it's not the case that we will
\emph{know} that the \acrshort{ads} is safe after it has driven $x$ kilometres on roads or
$y$ kilometres in a simulator -- we will never be able to predict all possible operating
situations in advance~\cite[1]{leahy2024grandchallenge}.
Therefore, we need to challenge the \acrshort{ads} as much as possible when testing it. Having it
take on a set of low-driveability test scenarios in advance of real-world operations,
we can be more confident in the correctness of our \acrshort{ads} if it is able to complete the
scenario. By making the scenario \emph{more complex} -- less driveable -- for
the \acrshort{ads}, our confidence in it will increase if it is able to complete the scenario. And
if it fails at executing the more complex scenario, we will potentially have uncovered an
underlying issue in the \acrshort{ads} that we did not know about so that we can fix it before it
causes harm in the real world.

Having an existing repository of \acrshort{ads} simulator scenarios, we wish to improve
them in such a way that they are less driveable and more challenging for the \acrshort{ads}.
\acrfullpl{llm} have demonstrated great capabilities of context learning and emergent
abilities~\cite[1]{LLM4AD}, which begs the question of their applicability for \acrshort{ads}
testing. We therefore ask: Can these existing test scenarios be made less driveable by applying
\acrshort{llm} technology to them?

\section{Problem description}\label{sec:problemDescription}

Traditional techniques for obtaining \acrshort{ads} scenarios rely on
\begin{inparaenum}
    \item highly skilled manual labour, or
    \item automated generation~\cite[1]{yao2025agentsllm}.
\end{inparaenum}
The prior incurs a significant cost, and is a major limitation in obtaining a large number of good
scenarios. The latter incurs a \emph{distributional shift} from the original scenarios, which can
undermine the validity of using them~\cite[1]{yao2025agentsllm}.

Moreover, even if we were to imagine a world in which we had infinite \begin{inparaenum}
    \item time and
    \item money, \end{inparaenum} we would not be able to successfully account for every possible
scenario.
There will always be more, unforeseen permutations of actors and actions. This is a reality
we need to deal with~\cite[1]{leahy2024grandchallenge}.
One possible measure of remedying with this, could be to \textit{decrease} the
driveability of our existing scenarios. Decreasing the driveability is not the same as suddenly
having access to the infinite set of possible scenarios, but it is reasonable to infer that being
able to \textit{test} the \acrshort{ads} (in a simulator) on these enhanced low-driveability
scenarios will leave it better fit for encountering other low-driveability scenarios in the wild
during operation. Having access to such a set of less driveable scenarios will allow \acrshort{ads}
operators to test their \acrshort{ads} that they assume to be working, and see if it still is able
to handle all these more challenging scenarios. If it is not, they will have gained a meaningful
insight into the workings of their \acrshort{ads} and can take action to remedy the fault before it
causes harm in the real world. And if the \acrshort{ads} \emph{does} still work, they can be more
confident in their system.

Finally, edge cases can be a major issue for \acrshort{ads} adaptation. The \textit{tail problem} as
it is known in the \acrfull{ml} field posits that \acrshort{ml} tasks are faced with a long tail of
unseen cases. We can map these unseen cases, to our unseen \acrshort{ads} scenarios. Because of
this, an \acrshort{ads} can be at risk of encountering an unseen edge case scenario during operation
-- something for which it might never have been tested~\cite[1]{yao2025agentsllm}.
Arguing that the \acrshort{ads} would probably crash simply due to it finding itself in an unseen
scenario is not logical. But it is important to keep in mind that the end we are pursing in the
broader adaption of \acrshort{ads}, is increased safety and efficiency on our roads. Not
sufficiently testing the \acrshort{ads} before deploying it would not serve our goal of increasing
road safety -- it would be a gamble with human lives. Not all edge cases are relevant for all
contexts and environments. As such, \acrshort{ads} system developers could employ \acrshortpl{llm}
to obtain more variants of a certain scenario that they wish to test, if they were to only be in possession of
a similar form. Say for example that they wished to obtain a set of scenarios in which
certain properties are met, or certain situations occur. 

Based on the problem description above, these are the formal research
questions of the thesis:

% Define each RQ as a macro so we can re-use them in the discussion. DRY.
\newcommand{\rqdecreasedriveability}{Can \acrlongpl{llm} be used to decrease the driveability of \acrlong{ads} simulator scenarios?}
\newcommand{\rqnohuman}{Is it feasible to employ \acrshortpl{llm} for obtaining unseen scenarios for \acrshort{ads} testing without human intervention?}

\begin{quote}
    \begin{questions}
        \item \rqdecreasedriveability
        \defrq{decrease-driveability}{\rqdecreasedriveability}\label{rq:decrease-driveability}
        \item \rqnohuman
        \defrq{no-human}{\rqnohuman}\label{rq:no-human}
    \end{questions}
\end{quote}

\newpage % Legger inn en newpage siden den nye siden uansett blir brukt opp.
% Dette ser bedre ut.

\section{Thesis overview}

Following this Introduction chapter, the thesis is structured as follows:

\begin{itemize}
    \item \Cref{chp:background} introduces key concepts related to \acrshortpl{ads} and \acrshortpl{llm}.
    \item \Cref{chp:relatedWorkAndLitReview} reviews the current state of research in the field and discusses related applied works and lessons learned.
    \item \Cref{chp:solutionProposal} details our proposed solution and its technical aspects.
    \item \Cref{chp:experiments} describes the experimental setup used to evaluate the solution.
    \item \Cref{chp:results} presents the findings from the experiments.
    \item \Cref{chp:discussion} analyses and contextualizes the results.
    \item \Cref{chp:furtherWork} suggests directions for future research.
    \item \Cref{chp:conclusion} summarizes the main contributions and findings.
\end{itemize}

Two appendices are included: \Nref{sec:fileDiffs} and \Nref{sec:errorMessages}.
