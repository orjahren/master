\chapter{Problem description}

\section{Cost}

Traditional techniques for obtaining \acrshort{ads} scenarios rely on high skilled manual labour.
This incurs a significant cost, and is a major limitation in obtaining a large number of good
scenarios, free from the bias of the author
% TODO: Bias er ikke relevnt her??


\section{Impossible to test all scenarios}
Furthermore, even if we were to imagine a world in which we had infinite \begin{inparaenum}
    \item time and
    \item money
\end{inparaenum}, we would not be able to successfully account for every possible scenario. This is
a reality we need to deal with. One possible measure of remedying with this, could be to
\textit{decrease} the driveability of our existing scenarios. Decreasing the driveability is not the
same as suddenly having access to the infinite set of possible scenarios, but it is reasonable to
infer that begin able to \textit{test} the \acrshort{ads} (in a simulator) on these enhanced
low-driveability scenarios will leave it better fit for encountering other low-driveability
scenarios in the wild during operation.

\section{Edge cases}

Edge cases can be a major issue for \acrshort{ads} adoptation. The \textit{tail problem} as it is
known in the \acrshort{ml} field posits that \acrshort{ml} tasks are faced with a long tail of
unseen cases. We can map these unseen cases, to our unseen \acrshort{ads} scenarios. Because of
this, an \acrshort{ads} can be at risk of encountering an unseen edge case scenario during
operation -- something for which it might never have been tested.
% TODO: Burde referere ting om tail problem? 
Arguing that the \acrshort{ads} would probably crash simply due to it finding itself in an unseen
scenario is not logical. But it is important to keep in mind that the end we are pursing in the
broader adaption of \acrfullpl{ads}, is increased saftey and efficiency on our roads. Not
sufficiently testing the \acrshort{ads} before deploying it would not serve our goal of increasing
road safety -- it would be a gamble with human lives.