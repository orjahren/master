% Farger/stil fra https://www.overleaf.com/learn/latex/Code_listing 
\definecolor{codegreen}{rgb}{0,0.6,0}
\definecolor{codegray}{rgb}{0.5,0.5,0.5}
\definecolor{codepurple}{rgb}{0.58,0,0.82}
\definecolor{backcolour}{rgb}{0.95,0.95,0.92}

\usepackage[rgb]{xcolor}%Nødvendig for UU-sjekk ifbm code listings. Uten dette blander den inn CMYK.

\lstdefinestyle{mystyle}{
  backgroundcolor=\color{backcolour},
  commentstyle=\color{codegreen},
  keywordstyle=\color{magenta},
  numberstyle=\tiny\color{codegray},
  stringstyle=\color{codepurple},
  basicstyle=\ttfamily\tiny,
  breakatwhitespace=false,
  breaklines=true,
  captionpos=b,
  keepspaces=true,
  numbers=left,
  numbersep=5pt,
  showspaces=false,
  showstringspaces=false,
  showtabs=false,
  tabsize=2
}

\lstset{style=mystyle}

% For å rendre diffs, fra https://tex.stackexchange.com/a/106129
\usepackage[svgnames]{xcolor}
\definecolor{diffstart}{named}{Grey}
\definecolor{diffincl}{named}{Green}
\definecolor{diffrem}{named}{OrangeRed}

\lstdefinelanguage{diff}{
  %basicstyle=\ttfamily\small,
  morecomment=[f][\color{diffstart}]{@@},
  morecomment=[f][\color{diffincl}]{+\ },
  morecomment=[f][\color{diffrem}]{-\ },
}

% Skrur av links i glossary-entries, se #5.
\glsdisablehyper{}
% \makeglossaries{}

% Set globals
\setglossarystyle{altlist}
\setdefaultenum{(1)}{(a)}{i.}{A.}
\interfootnotelinepenalty=10000

\setlength\epigraphwidth{.8\textwidth}
\setlength\epigraphrule{0pt}