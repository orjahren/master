
\subsection{\acrfull{ml}}

% \tanke{Broad background of what is \acrshort{ml}}

\acrlong{ml} entails having a machine learn from experience. The way they can do this, is by
learning from data~\cite[4]{marsland}. The main advantages of using a computer system that has
learnt is that it will be capable of \textit{remembering}, \textit{adapting}, and
\textit{generalising} its experience to new situations~\cite[4]{marsland}, allowing it to solve
new problems that it might not have been specifically trained to handle.
% Denne siste biten er fin fordi det tenkes å være relevant for selvkjørende biler...kan ikke trene
% for alt.

\subsubsection*{Types of \acrlong{ml}}

\acrlong{ml} is a somewhat general concept [citation needed], that is instantiated in various ways.
Some of the more popular ways of implementing \acrshort{ml} include Random Forest, Decision trees,
and KNN.

\todo{Can i say instantiated like this?}

\acrshort{ml} is divided in 2 principal categories: supervised learning, and unsupervised learning.
Their main difference is whether the input data that the model will use for training is labelled. If
the data is un-labelled, the \acrshort{ml} algorithm will find features on its own.

\tanke{Burde forklare hva labelled betyr? Og features?}

\subsubsection*{Labelled and unlabelled data}

Data is labelled if it has a given category assigned to it. If the datum is unlabelled, our system
has no \textit{a priori} knowledge of the different data.

\subsubsection*{Features in \acrlong{ml}}

Features are what we use for learning.



\subsubsection{\acrfull{dl}}

% \tanke{Broad background of what is DL and how it differs from general \acrshort{ml}}

\acrlong{dl} is one form of \acrlong{ml}. It utilises \textit{neural networks}.

\subsubsection{Metrics}

Present general \acrshort{ml} metrics such as Precision, recall, f1-score etc

\tanke{This does not seem to relevant for the theis - unclear usage of these metrics in the project}