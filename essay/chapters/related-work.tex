\section{Related work}\label{sec:relatedWork}

\todo{What subsections deserve to be included here?}

\subsection{DeepScenario}\label{sec:deepScenario}

DeepScenario is both a dataset and a toolset aimed at \acrlong{ad} testing~\cite{DeepScenario}. It
builds on the work of~\cite{CriticalScenarios} and provides executable test situations that are
applicable for the simulator SVL by LG~\cite{lgsvl}.

The project utilises traditional machine learning methodologies, having been performed prior to the
broad diffusion of \acrshort{llms}.

Our project recognizes a significant knowledge gap in building upon this work using \acrshort{llm} technology.

\subsection{RTCM}
RTCM is a \acrshort{ad} testing framework that allows the user to utilise natural language for
synthezising test cases~\cite{RTCM}.

Having utilised natural language for \acrshort{ad} testing before the advent of \acrshort{llms},
RTCM is in some ways a forerunner for this project.

\subsection{AutoSceneGen}

AutoSceneGen is a framework for \acrshort{ad} testing using \acrshort{llms},
focusing on the motion planning of \acrlong{ads}~\cite[14539]{autoSceneGen}.
\citeauthor{autoSceneGen} highlights how \acrshort{llms} provide opportunities
for efficiently evaluating \acrshort{ads} in a cost-effective manner~\cite[14539-14540]{autoSceneGen}.
\tanke{Jeg vil også fokusere på motion planning??}


\subsection{LLM4AD}

LLM4AD is a paper that gives a broad overview of \acrshort{llms} for \acrlong{ads}. It touches on
several of the various \acrshort{ad} applications where \acrshort{llms} are relevant~\cite{LLM4AD}.